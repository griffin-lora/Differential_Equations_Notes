\documentclass[notes]{subfiles}

\begin{document}

\setcounter{section}{20}
\section{Lecture (4/23/25)}

\subsection{Minimum Radius of Convergence for Power Series Solutions}
\begin{theorem}
    Given an order two linear ODE in the form $y'' + P(x)y' + Q(x)y = f(x)$ then the minimum radius of convergence for a power series solution to the ODE centered at $a$ is the distance between $a$ and the nearest singular point.
\end{theorem}

\begin{exercise}
    Find the minimum radius of convergence for a power series solution centered at $7$ for $(x^2 - 9)y'' + xy' + 3(x + 4)y = 0$.
\end{exercise}
\begin{solution}
    First we put the ODE in standard form, so
    \begin{align*}
        (x^2 - 9)y'' + xy' + 3(x + 4)y = 0
        &\iff y'' + \frac{x}{(x - 3)(x + 3)}y' + \frac{3(x + 4)}{(x - 3)(x + 3)} = 0
    \end{align*}
    Therefore, the singular points are $\pm 3$, so the minimum radius of convergence is $|7 - 3| = 4$.
\end{solution}

\subsection{Method of Frobenius}
However, we often want to center our solution at singular points.

\begin{definition}
    A singular point $x_0$ for a second order linear ODE $y'' + P(x)y' + Q(x)y = 0$ is \textsl{regular} iff $(x \mapsto (x - x_0)P(x))$ and $(x \mapsto (x - x_0)^2 Q(x))$ are analytic at $x_0$.
\end{definition}

The method of Frobenius lets us solve at regular singular points. However, can't really do much of anything at irregular singular points.

\begin{exercise}
    Find and classify singular points for $(x^2 - 4)y'' + 5(x - 2)y' + 3y = 0$.
\end{exercise}
\begin{solution}
    First we put the ODE in standard form, so
    \begin{align*}
        (x^2 - 4)y'' + 5(x - 2)y' + 3y = 0
        &\iff y'' + \frac{5(x - 2)}{(x - 2)(x + 2)}y' + \frac{3}{(x - 2)(x + 2)}y = 0 \\
        &\iff y'' + \frac{5}{x + 2}y' + \frac{3}{(x - 2)(x + 2)}y = 0
    \end{align*}
    Therefore, we have singular points of $\pm 2$. Now consider the singular point $-2$, so we get
    \[
        \frac{5(x + 2)}{x + 2} = 5 \quad \text{and} \quad \frac{3(x + 2)^2}{(x - 2)(x + 2)} = \frac{3(x + 2)}{x - 2}
    \]
    Therefore, $-2$ is a regular singular point.
    Considering $2$ we get
    \[
        \frac{5(x - 2)}{x + 2} \quad \text{and} \quad \frac{3(x - 2)^2}{(x - 2)(x + 2)} = \frac{3(x - 2)}{x + 2}
    \]
    Therefore, $2$ is also a regular singular point.
\end{solution}

\begin{exercise}
    Find and classify singular ponts for $(x^2 - 4)^2y'' + 5(x - 2)y' + 3y = 0$.
\end{exercise}
\begin{solution}
    First we put the ODE in standard form, so
    \begin{align*}
        (x^2 - 4)y'' + 5(x - 2)y' + 3y = 0
        &\iff y'' + \frac{5(x - 2)}{(x - 2)^2(x + 2)^2}y' + \frac{3}{(x - 2)^2(x + 2)^2}y = 0 \\
        &\iff y'' + \frac{5}{(x - 2)(x + 2)^2}y' + \frac{3}{(x - 2)^2(x + 2)^2}y = 0
    \end{align*}
    Considering $-2$ we get
    \[
        \frac{5(x + 2)}{(x - 2)(x + 2)^2} = \frac{5}{(x - 2)(x + 2)}
    \]
    But this is not analytic at $-2$, so $-2$ is an irregular singular point.
    Considering $2$ we get
    \[
        \frac{5(x - 2)}{(x - 2)(x + 2)^2} = \frac{5}{(x + 2)^2} \quad \text{and} \quad \frac{3(x - 2)^2}{(x - 2)^2(x + 2)^2} = \frac{3}{(x + 2)^2}
    \]
    Which is analytic at $2$ so $2$ is a regular singular point.
\end{solution}

\begin{theorem}[Method of Frobenius]
    If $x_0$ is a regular singular point for $y'' + P(x)y' + Q(x)y = 0$ then there exists a solution of the form
    \[
        y = (x - x_0)^r \sum_{n = 0}^\infty c_n(x - x_0) = \sum_{n = 0}^\infty c_n(x - x_0)^{n + r}
    \]
    for some $r \in \real$, called an indicial root.
\end{theorem}

Since we don't have a constant term when differentiating for $r \notin \mathbb{Z}$ then we don't move the first index.

\begin{exercise}
    Solve the ODE $3xy'' + y' - y = 0$ about $0$.
\end{exercise}
\begin{solution}
    In standard form we get
    \begin{align*}
        3xy'' + y' - y = 0
        &\iff y'' + \frac{y'}{3x} - \frac{y}{3x} = 0
    \end{align*}
    $0$ is a regular singular point, so consider
    \begin{align*}
        y = \sum_{n = 0}^\infty c_n x^{n + r}
        &\iff y' = \sum_{n = 0}^\infty (n + r)c_n x^{n + r - 1}
        \iff y'' = \sum_{n = 0}^\infty (n + r)(n + r - 1)c_n x^{n + r - 2}
    \end{align*}
    Now we build to get
    \begin{align*}
        0
        &= 3xy'' + y' - y \\
        &= 3x\sum_{n = 0}^\infty (n + r)(n + r - 1)c_n x^{n + r - 2} + \sum_{n = 0}^\infty (n + r)c_n x^{n + r - 1} - \sum_{n = 0}^\infty c_n x^{n + r} \\
        &= \sum_{n = 0}^\infty 3(n + r)(n + r - 1)c_n x^{n + r - 1} + \sum_{n = 0}^\infty (n + r)c_n x^{n + r - 1} - \sum_{n = 0}^\infty c_n x^{n + r} \\
        &= \sum_{n = -1}^\infty 3(n + r + 1)(n + r)c_{n + 1} x^{n + r} + \sum_{n = -1}^\infty (n + r + 1)c_{n + 1} x^{n + r} - \sum_{n = 0}^\infty c_n x^{n + r} \\
        &= 3r(r - 1)c_0x^{r - 1} + rc_0x^{r - 1} + \sum_{n = 0}^\infty ((3(n + r + 1)(n + r) + n + r + 1)c_{n + 1} - c_n)x^{n + r} \\
        &= (3r(r - 1) + r)c_0x^{r - 1} + \sum_{n = 0}^\infty ((3(n + r + 1)(n + r) + n + r + 1)c_{n + 1} - c_n)x^{n + r}
    \end{align*}
    Now consider that
    \begin{align*}
        0 = (3r(r - 1) + r)c_0
        &\iff 0 = 3r(r - 1) + r = 3r^2 - 2r = r(3r - 2) \\
        &\iff r = 0, r = \frac{2}{3}
    \end{align*}
    and
    \begin{align*}
        &0 = (3(n + r + 1)(n + r) + n + r + 1)c_{n + 1} - c_n \quad \text{for $n \geq 0$} \\
        \iff& c_{n + 1} = \frac{c_n}{(n + r + 1)(3n + 3r + 1)}
    \end{align*}
    Considering $r = 0$ we get
    \begin{align*}
        c_{n + 1} &= \frac{c_n}{(n + 1)(3n + 1)} \\
        n = 0 &\implies c_1 = \frac{c_0}{1\cdot 1} \\
        n = 1 &\implies c_2 = \frac{c_1}{2\cdot 4} = \frac{c_0}{1\cdot 2\cdot 1\cdot 4} \\
        n = 2 &\implies c_3 = \frac{c_2}{3\cdot 7} = \frac{c_0}{1\cdot 2\cdot 3\cdot 1\cdot 4\cdot 7} \\
        n = 3 &\implies c_4 = \frac{c_3}{4\cdot 10} = \frac{c_0}{1\cdot 2\cdot 3\cdot 4\cdot 1\cdot 4\cdot 7\cdot 10}
    \end{align*}
    Therefore, $c_n = \frac{c_0}{n!((3n - 2)!!!)}$, so $y_1 = \sum_{n = 0}^\infty \frac{c_0}{n!((3n - 2)!!!)}x^n$. \\
    Considering $r = \frac{2}{3}$ we get
    \begin{align*}
        c_{n + 1} &= \frac{c_n}{(n + \frac{5}{3})(3n + 3)} = \frac{c_n}{(3n + 5)(n + 1)} \\
        n = 0 &\implies c_1 = \frac{c_0}{5\cdot 1} \\
        n = 1 &\implies c_2 = \frac{c_1}{8\cdot 2} = \frac{c_0}{1\cdot 2\cdot 5\cdot 8} \\
        n = 2 &\implies c_3 = \frac{c_0}{1\cdot 2\cdot 3\cdot 5\cdot 5\cdot 8\cdot 11} \\
        n = 3 &\implies c_4 = \frac{c_0}{1\cdot 2\cdot 3\cdot 4\cdot 5\cdot 5\cdot 8\cdot 11\cdot 14}
    \end{align*}
    Therefore, we have that
    \[
        y_2 = c_0x^{\frac{2}{3}}\left( 1 + \frac{1}{5\cdot 1}x + \frac{1}{1\cdot 2\cdot 5\cdot 8}x^2 + \frac{1}{1\cdot 2\cdot 3\cdot 5\cdot 5\cdot 8\cdot 11}x^3 + \frac{1}{1\cdot 2\cdot 3\cdot 4\cdot 5\cdot 5\cdot 8\cdot 11\cdot 14}x^4 + \ldots \right)
    \]
\end{solution}

\end{document}