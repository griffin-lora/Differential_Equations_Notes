\documentclass[notes]{subfiles}
\externaldocument{lecture_11}

\begin{document}

\setcounter{section}{13}
\section{Lecture (3/5/25)}

\subsection{Cauchy-Euler Equations}
\begin{definition}[Cauchy-Euler Equation]
    A linear differential equation is a \textsl{Cauchy-Euler equation} iff it is of the form
    \[
        \sum_{i = 0}^n a_i x^i \frac{d^i y}{dx^i} = g(x)
    \]
    where $0^0 = 1$
\end{definition}
These are essentially constant coefficient linear differential equations with corresponding powers of $x$.

\begin{example}
    $ax^2 y'' + bxy + c = 0$ is a second order homogenous Cauchy-Euler equation.
\end{example}

To solve these, we make a guess of $y = x^m$. Therefore our basis solutions will be of the form $y_i = x^{m_i}$ where $m_i$ is the root of the resulting characteristic polynomial from our guess.

\begin{exercise}
    Solve $x^2y'' - 2xy' - 4y = 0$.
\end{exercise}
\begin{solution}
    Since this equation is Cauchy-Euler, so we guess $y = x^m$, then $y' = mx^{m - 1}$ and $y'' = m(m - 1)x^{m - 2}$.
    Now we build the ODE so
    \begin{align*}
        0
        &= x^2y'' - 2xy' - 4y
        = m(m - 1)x^2x^{m - 2} - 2mxx^{m - 1} - 4x^m \\
        &= m(m - 1)x^m - 2mx^m - 4x^m
    \end{align*}
    If $x = 0$, then the leading term of the ODE would vanish, so it would not be a part of the interval of validity. Therefore, we can say that $x \neq 0$, which implies that $x^m \neq 0$. We now divide through by $x^m$ to get the characteristic polynomial so
    \begin{align*}
        &0 = m(m - 1) - 2m - 4 = m^2 - m - 2m - 4 = m^2 - 3m - 4 = (m - 4)(m + 1) \\
        \iff& m = -1, m = 4
    \end{align*}
    Therefore $y = c_1x^{-1} + c_2x^4$ solves the ODE.
\end{solution}

We will now consider the case where we have repeated roots

\begin{exercise}
    Solve $4x^2y'' + 8xy' + y = 0$.
\end{exercise}
\begin{solution}
    Consider the characteristic polynomial so
    \begin{align*}
        &0 = 4m(m - 1) + 8m + 1 = 4m^2 - 4m + 8m + 1 = 4m^2 + 4m + 1 = (2m + 1)^2 \\
        \iff& 2m + 1 = 0
        \iff m = -\frac{1}{2}
    \end{align*}
    Therefore we get a basis solution of $y_1 = x^{-\frac{1}{2}}$. However, we need $y_2$ so we use reduction of order. We have to put in standard form first so the equation is $y'' + \frac{2}{x}y' + \frac{1}{4x^2}y = 0$. We also impose that $x > 0$, then
    \begin{align*}
        y_2
        &= y_1 \int \frac{e^{-\int P(x)dx}}{y_1^2}dx
        = x^{-\frac{1}{2}} \int \frac{e^{-2\int \frac{1}{x}dx}}{x^{-1}}dx
        = x^{-\frac{1}{2}} \int \frac{e^{-2\ln x}}{x^{-1}}dx \\
        &= x^{-\frac{1}{2}} \int \frac{x^{-2}}{x^{-1}}dx
        = x^{-\frac{1}{2}} \int x^{-1}dx
        = x^{-\frac{1}{2}}\ln x
    \end{align*}
    Therefore $y = c_1x^{-\frac{1}{2}} + c_2x^{-\frac{1}{2}}\ln x$ solves the ODE.
\end{solution}

There are three cases for the characteristic polynomial

\textbf{Case 1}: We get a distinct real root $m_k$. Then the basis solution looks like
\[
    y_k = x^{m_k}
\]

\textbf{Case 2}: We get real root $m_k$ of multiplicity $p$. Then the basis solutions look like
\[
    [y_{k, j} = x^{m_k}(\ln x)^{j - 1}]_{j = 1}^p
\]

\textbf{Case 3}: We get a pair of complex conjugate roots $m_k = \alpha \pm i\beta$. Then the basis solutions look like
\[
    y_{k, 1} = x^\alpha \cos(\beta \ln x) \quad \text{and} \quad y_{k, 2} = x^\alpha \sin(\beta \ln x)
\]

\begin{exercise}
    Solve $x^3 y''' + 5x^2 y'' + 7xy' + 8y = 0$.
\end{exercise}
\begin{solution}
    Consider the characteristic polynomial so
    \begin{align*}
        0
        &= m(m - 1)(m - 2) + 5m(m - 1) + 7m + 8 \\
        &= m(m^2 - 3m + 2) + 5m(m - 1) + 7m + 8 \\
        &= m^3 - 3m^2 + 2m + 5m^2 - 5m + 7m + 8 \\
        &= m^3 + 2m^2 + 4m + 8
        = m^2(m + 2) + 4(m + 2) \\
        &= (m^2 + 4)(m + 2)
        = (m - 2i)(m + 2i)(m + 2)
    \end{align*}
    Therefore $m = \pm 2i$ or $m = -2$ so $y = c_1x^{-2} + c_2\cos(2\ln x) + c_3\sin(2\ln x)$ solves the ODE.
\end{solution}

\subsection{Inhomogeneous Cauchy-Euler Equations}
If we have an inhomogeneous Cauchy-Euler equation, then we cannot solve it by merely guessing a particular solution or using annihilator operators. The only method that will work is variation of parameters.

\begin{exercise}
    Solve $x^2y'' - 3xy' + 3y = 2x^4e^x$.
\end{exercise}
\begin{solution}
    Consider the characteristic polynomial so
    \begin{align*}
        &0 = m(m - 1) - 3m + 3 = m^2 - m - 3m + 3 = m^2 - 4m + 3 = (m - 1)(m - 3) \\
        \iff& m = 1, m = 3
    \end{align*}
    Therefore the basis solutions are $y_1 = x$ and $y_2 = x^3$.
    
    To use variation of parameters we must consider the standard form so $y'' - \frac{3}{x}y' + \frac{3}{x^2} = 2x^2e^x$. We guess $y_p = u_1y_1 + u_2y_2$. We now consider the Wro\'nskian so
    \begin{align*}
        W(y_1, y_2)
        &= \begin{vmatrix}
            x & x^3 \\
            1 & 3x^2
        \end{vmatrix}
        = 3x^3 - x^3 = 2x^3
    \end{align*}
    Now by \cref{var_params} we get
    \begin{align*}
        u_1
        &= \int \frac{-y_2f(x)}{W(y_1, y_2)} dx
        = \int \frac{-x^3(2x^2 e^x)}{2x^3} dx
        = -\int x^2e^x dx
    \end{align*}
    Now we do integration by parts so $v = x^2 \iff dv = 2xdx$ and $dz = e^xdx \iff z = e^x$ therefore
    \begin{align*}
        u_1
        &= -\left( x^2e^x - 2\int xe^x dx \right)
    \end{align*}
    Now we do a second integration by parts so $w = x \iff dw = dx$ therefore
    \begin{align*}
        u_1
        &= -\left( x^2e^x - 2\left( xe^x - \int e^x dx \right) \right) \\
        &= -\left( x^2e^x - 2\left( xe^x - e^x \right) \right) \\
        &= -x^2e^x + 2\left( xe^x - e^x \right)
        = -x^2e^x + 2xe^x - 2e^x
    \end{align*}
    For $u_2$ we get
    \begin{align*}
        u_2
        &= \int \frac{y_1f(x)}{W(y_1, y_2)} dx
        = \int \frac{2x^3e^x}{2x^3} dx
        = \int e^x dx
        = e^x
    \end{align*}
    Therefore
    \begin{align*}
        y_p
        &= x(-x^2e^x + 2xe^x - 2e^x) + x^3e^x
        = -x^3e^x + 2x^2e^x - 2xe^x + x^3e^x \\
        &= 2x^2e^x - 2xe^x
    \end{align*}
    So now we have that $y = c_1x + c_2x^3 + 2x^2e^x - 2xe^x$ solves the ODE.
\end{solution}

We will now show that Cauchy-Euler equations are really just constant coefficient linear ODEs.

\begin{theorem}
    Given a Cauchy-Euler equation in the form
    \[
        \sum_{i = 0}^n a_i x^i \frac{d^iy}{dx^i} = g(x)
    \]
    where $x > 0$, then the substitution $x = e^t$ will give a constant coefficient linear ODE in terms of $y$ and $t$.
\end{theorem}

\begin{theorem}
    Given a second-order Cauchy-Euler equation in the form
    \[
        ax^2\frac{d^2y}{dx^2} + bx\frac{dy}{dx} + cy = g(x)
    \]
    where $x > 0$, then the substitution $x = e^t$ will result in the equation
    \[
        a\frac{d^2y}{dt^2} + (-a + b)\frac{dy}{dt} + cy = g(e^t)
    \]
\end{theorem}
\begin{proof}
    Consider $x = e^t \iff t = \ln x$.
    Now we consider the second term so
    \begin{align*}
        \frac{dy}{dx}
        &= \frac{dy}{dt}\frac{dt}{dx}
        = \frac{1}{x}\frac{dy}{dt}
    \end{align*}
    Considering the leading term we get
    \begin{align*}
        \frac{d^2y}{dx^2}
        &= \frac{d}{dx}\left( \frac{dy}{dx} \right)
        = \frac{d}{dx}\left(\frac{dy}{dt}\frac{dt}{dx} \right)
        = \frac{dt}{dx}\frac{d}{dx}\left( \frac{dy}{dt} \right) + \frac{dy}{dt}\frac{d}{dx}\left( \frac{dt}{dx} \right) \\
        &= \frac{dt}{dx}\frac{d^2y}{dxdt} + \frac{dy}{dt}\frac{d^2t}{dx^2}
        = \frac{dt}{dx}\frac{d^2y}{dxdt}\frac{dt}{dt} + \frac{dy}{dt}\frac{d^2t}{dx^2}
        = \frac{dt}{dx}\frac{d^2y}{dt^2}\frac{dt}{dx} + \frac{dy}{dt}\frac{d^2t}{dx^2} \\
        &= \left(\frac{dt}{dx}\right)^2\frac{d^2y}{dt^2} + \frac{d^2t}{dx^2}\frac{dy}{dt}
        = \frac{1}{x^2}\frac{d^2y}{dt^2} - \frac{1}{x^2}\frac{dy}{dt}
        = \frac{1}{x^2} \left( \frac{d^2y}{dt^2} - \frac{dy}{dt} \right)
    \end{align*}
    If we build the ODE with these then we get
    \begin{align*}
        g(e^t)
        &= ax^2\left(\frac{1}{x^2}\left( \frac{d^2y}{dt^2} - \frac{dy}{dt} \right)\right) + bx\left(\frac{1}{x}\frac{dy}{dt}\right) + cy
        = a\frac{d^2y}{dt^2} - a\frac{dy}{dt} + b\frac{dy}{dt} + cy \\
        &= a\frac{d^2y}{dt^2} + (-a + b)\frac{dy}{dt} + cy
    \end{align*}
\end{proof}

\end{document}