\documentclass[notes]{subfiles}

\begin{document}

\setcounter{section}{12}
\section{Lecture (3/3/25)}

\subsection{Annihilation Procedure}
\begin{procedure}[Solving by annihilation]
    ~

    \begin{enumerate}[label = (\arabic*)]
        \item Solve for the basis solutions
        \item Write in operator form
        \item Find annihilator for driver
        \item Apply annihilator
        \item Solve homogeneous ODE
        \item Pick out new solutions
        \item Find constants for particular solution
    \end{enumerate}
\end{procedure}

\begin{exercise}
    Solve $y'' + y = x\cos x - \cos x$ by annihilation.
\end{exercise}
\begin{solution}
    Consider the characteristic polynomial of the associated homogeneous ODE so
    \begin{align*}
        0 = m^2 + 1 = (m - i)(m + 1)
        &\iff m = \pm i
    \end{align*}
    Therefore $y_n = c_1 \cos x + c_2 \sin x$. If we weren't solving by annihilation we could consider $y_p = (Ax + B)\cos x$ but then it would be linearly dependent with the basis solutions. So we would guess $y_p = (Ax^2 + Bx)\cos x$.
    
    For solving by annihilation, let $g(x) = x\cos x - \cos x$ so the operator form is $D^2 + 1 = g$.

    Notice that $D^2 + 1$ annihilates just the second term of $g$. By reduction of order, we can simply bump the multiplicity up to get $(D^2 + 1)^2$ as an annihilator for $g$, therefore
    \begin{align*}
        D^2 + 1 = g
        &\iff (D^2 + 1)^2(D^2 + 1) = (D^2 + 1)^2 g \\
        &\iff (D^2 + 1)^3 = 0
    \end{align*}
    Therefore we can consider the characteristic polynomial so
    \begin{align*}
        0 = (m^2 + 1)^3 = (m - i)^3(m + i)^3
        &\iff m = \pm i
    \end{align*}
    Therefore since we have multiplicity $3$ then the solution to the annihilated ODE is
    \[
        y = c_1\cos x + c_2\sin x + c_3x\cos x + c_4x\sin x + c_5x^2\cos x + c_6x^2\sin x
    \]
    Since we have gained $4$ new terms then the particular solution must be
    \[
        y_p = c_3x\cos x + c_4x\sin x + c_5x^2\cos x + c_6x^2\sin x
    \]
    Relabelling and factoring we get the same as the guess so
    \[
        y_p = (Ax^2 + Bx)\cos x + (Cx^2 + Dx)\sin x
    \]
    Now we find derivatives
    \begin{align*}
        y_p'
        &= (2Ax + B)\cos x - (Ax^2 + Bx)\sin x + (2Cx + D)\sin x + (Cx^2 + Dx)\cos x \\
        &= (Cx^2 + (2A + D)x + B)\cos x + (-Ax^2 + (-B + 2C)x + D)\sin x
    \end{align*}
    and
    \begin{align*}
        y_p''
        &= (2Cx + 2A + D)\cos x - (Cx^2 + (2A + D)x + B)\sin x \\ &+ (-2Ax - B + 2C)\sin x + (-Ax^2 + (-B + 2C)x + D)\cos x \\
        &= (-Ax^2 + (-B + 4C)x + 2A + 2D)\cos x + (-Cx^2 + (-4A - D)x - 2B + 2C)\sin x
    \end{align*}
    And build the ODE
    \begin{align*}
        x\cos x - \cos x
        &= y_p'' + y_p \\
        &= (-Ax^2 + (-B + 4C)x + 2A + 2D)\cos x \\
        &+ (-Cx^2 + (-4A - D)x - 2B + 2C)\sin x \\
        &+ (Ax^2 + Bx)\cos x + (Cx^2 + Dx)\sin x \\
        &= (4Cx + 2A + 2D)\cos x + (-4Ax - 2B + 2C)\sin x
    \end{align*}
    iff
    \begin{align*}
        &(4Cx + 2A + 2D)\cos x + (-4Ax - 2B + 2C)\sin x = (x - 1)(\cos x) \\
        \iff& 4C = 1 \iff C = \frac{1}{4} \\
        \iff& -4A = 0 \iff A = 0 \\
        \iff& 2A + 2D = -1 \iff 2D = -1 \iff D = -\frac{1}{2} \\
        \iff& -2B + 2C = 0 \iff B = C = \frac{1}{4}
    \end{align*}
    So the particular solution is $y_p = \frac{1}{4}x\cos x + \left( \frac{1}{4}x^2 - \frac{1}{2}x \right)\sin x$. Therefore
    \[
        y = c_1 \cos x + c_2 \sin x + \frac{1}{4}x\cos x + \left( \frac{1}{4}x^2 - \frac{1}{2}x \right)\sin x
    \]
    solves the ODE.
\end{solution}

\subsection{Variation of Parameters}
Consider the case where the driver function is not a nice function in terms of exponentials or polynomials. This is where variation of parameters is useful. This is essentially a formula for a particular solution in terms of the coefficients, driver function, and basis solution.

Given an order two linear ODE in the form $y'' + P(x)y' + Q(x)y = f(x)$, then it has basis solutions $y_1$ and $y_2$. Now assume that $y_p = u_1y_1 + u_2y_2$ where $u_1$ and $u_2$ depend on $x$.

Now we find derivatives so
\begin{align*}
    y_p' &= u_1'y_1 + u_1y_1' + u_2'y_2 + u_2y_2'
\end{align*}
and
\begin{align*}
    y_p''
    &= u_1''y_1 + u_1'y_1' + u_1'y_1' + u_1y_1'' + u_2''y_2 + u_2'y_2' + u_2'y_2' + u_2y_2'' \\
    &= u_1''y_1 + 2u_1'y_1' + u_1y_1'' + u_2''y_2 + 2u_2'y_2' + u_2y_2''
\end{align*}
Building the ODE we get
\begin{align*}
    f(x)
    &= y_p'' + P(x)y_p' + Q(x)y_p \\
    &= u_1''y_1 + 2u_1'y_1' + u_1y_1'' + u_2''y_2 + 2u_2'y_2' + u_2y_2'' \\ &+ P(x)(u_1'y_1 + u_1y_1' + u_2'y_2 + u_2y_2') + Q(x)(u_1y_1 + u_2y_2) \\
    &= u_1(y_1'' + P(x)y_1' + Q(x)y_1) + u_2(y_2'' + P(x)y_2' + Q(x)y_2) \\ &+ u_1''y_1 + 2u_1'y_1' + u_2''y_2 + 2u_2'y_2' + P(x)(u_1'y_1 + u_2'y_2) \\
    &= u_1''y_1 + 2u_1'y_1' + u_2''y_2 + 2u_2'y_2' + P(x)(u_1'y_1 + u_2'y_2) \\
    &= u_1''y_1 + u_1'y_1' + u_2''y_2 + u_2'y_2' + P(x)(u_1'y_1 + u_2'y_2) + u_1'y_1' + u_2'y_2' \\
    &= \frac{d}{dx}(u_1'y_1 + u_2'y_2) + P(x)(u_1'y_1 + u_2'y_2) + u_1'y_1' + u_2'y_2'
\end{align*}
By black magic we now make an assumption that $u_1'y_1 + u_2'y_2 = 0$, so $u_1'y_1' + u_2'y_2' = f(x)$. Treating everything pointwise with $y_1, y_2$ as coefficients and $u_1', u_2'$ as unknowns, we can now construct an augmented coefficient matrix for this linear system of equations so
\[
    \begin{amatrix}{2}
        y_1 & y_2 & 0 \\
        y_1' & y_2' & f(x)
    \end{amatrix}
\]
By Cramer's rule then
\begin{align*}
    u_1'
    &= \frac{\begin{vmatrix}
        0 & y_2 \\
        f(x) & y_2'
    \end{vmatrix}}{\begin{vmatrix}
        y_1 & y_2 \\
        y_1' & y_2'
    \end{vmatrix}}
    = \frac{-y_2f(x)}{\begin{vmatrix}
        y_1 & y_2 \\
        y_1' & y_2'
    \end{vmatrix}}
\end{align*}
\begin{align*}
    u_2'
    &= \frac{\begin{vmatrix}
        y_1 & 0 \\
        y_1' & f(x)
    \end{vmatrix}}{\begin{vmatrix}
        y_1 & y_2 \\
        y_1' & y_2'
    \end{vmatrix}}
    = \frac{y_1f(x)}{\begin{vmatrix}
        y_1 & y_2 \\
        y_1' & y_2'
    \end{vmatrix}}
\end{align*}
We can now rewrite in terms of the Wro\'nskian so we get
\begin{equation} \label{var_params}
    u_1' = \frac{-y_2f(x)}{W(y_1, y_2)} \quad \text{and} \quad u_2' = \frac{y_1f(x)}{W(y_1, y_2)}
\end{equation}

\begin{exercise}
    Solve $4y'' + 36y = \csc(3x)$.
\end{exercise}
\begin{solution}
    For the characteristic polynomial of the homogeneous ODE we get
    \begin{align*}
        0 = 4m^2 + 36
        &\iff 0 = m^2 + 9 = (m - 3i)(m + 3i)
        \iff m = \pm 3i
    \end{align*}
    Therefore the basis solutions are $y_1 = \cos(3x)$ and $y_2 = \sin(3x)$.
    
    Now we assume $y_p = u_1y_1 + u_2y_2$. To get $f$ we rewrite as $y'' + 9y = \frac{1}{4}\csc(3x)$. Now we need the Wro\'nskian so
    \begin{align*}
        W(y_1, y_2)
        &= \begin{vmatrix}
            \cos(3x) & \sin(3x) \\
            -3\sin(3x) & 3\cos(3x)
        \end{vmatrix}
        = 3\cos^2(3x) + 3\sin^2(3x)
        = 3
    \end{align*}
    Therefore by \eqref{var_params} we get
    \begin{align*}
        u_1
        &= \int \frac{-y_2f(x)}{W(y_1, y_2)} dx
        = \int \frac{-\sin(3x)\csc(3x)}{12} dx \\
        &= -\frac{1}{12} \int \sin(3x)\csc(3x) dx
        = -\frac{1}{12} \int dx
        = -\frac{1}{12}x
    \end{align*}
    and
    \begin{align*}
        u_2
        &= \int \frac{y_1f(x)}{W(y_1, y_2)} dx
        = \int \frac{\cos(3x)\csc(3x)}{12} dx \\
        &= \frac{1}{36} \int \frac{3\cos(3x)}{\sin(3x)} dx
    \end{align*}
    Now let $v = \sin(3x)$ so $dv = 3\cos(3x)dx$ therefore
    \begin{align*}
        u_2
        &= \frac{1}{36} \int \frac{1}{v} dv
        = \frac{1}{36}\ln|v|
        = \frac{1}{36}\ln\left|\sin(3x)\right|
    \end{align*}
    So the particular solution is $y_p = -\frac{1}{12}x\cos(3x) + \frac{1}{36}\ln\left|\sin(3x)\right|\sin(3x)$. Therefore
    \[
        y = c_1 \cos(3x) + c_2\sin(3x) - \frac{1}{12}x\cos(3x) + \frac{1}{36}\ln\left|\sin(3x)\right|\sin(3x)
    \]
    solves the ODE.
\end{solution}

\end{document}