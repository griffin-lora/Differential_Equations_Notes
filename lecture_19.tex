\documentclass[notes]{subfiles}

\begin{document}

\setcounter{section}{20}
\section{Lecture (4/14/25)}

\subsection{Differentiation in The s-Domain}
\begin{theorem}
    \[
        \left( \frac{d}{ds} \right)^n (\lap f(s)) = (-1)^n\lap(t^n f(t))
    \]
\end{theorem}
\begin{proof}
    First consider
    \begin{align*}
        \left( \frac{d}{ds} \right)^n (\lap f(s))
        &= \left( \frac{d}{ds} \right)^n \int_0^\infty f(t)e^{-st}dt
        = \int_0^\infty \left( \frac{\del}{\del s} \right)^n f(t)e^{-st}dt
    \end{align*}
    We now want to show by induction that
    \[
        \left( \frac{\del}{\del s} \right)^n f(t)e^{-st} = (-1)^n t^n f(t)e^{-st}
    \]
    For the base case of $n = 0$ then
    \begin{align*}
        \left( \frac{\del}{\del s} \right)^0 f(t)e^{-st}
        &= f(t)e^{-st}
        = (-1)^0 t^0 f(t)e^{-st}
    \end{align*}
    Now we assume our induction hypothesis and consider
    \begin{align*}
        \left( \frac{\del}{\del s} \right)^{n + 1} f(t)e^{-st}
        &= \frac{\del}{\del s} \left( \frac{\del}{\del s} \right)^n f(t)e^{-st}
        = \frac{\del}{\del s} (-1)^n t^n f(t)e^{-st} \\
        &= (-1)^{n + 1} t^{n + 1} f(t)e^{-st}
    \end{align*}
    Therefore
    \begin{align*}
        \left( \frac{d}{ds} \right)^n (\lap f(s))
        &= \int_0^\infty \left( \frac{\del}{\del s} \right)^n f(t)e^{-st}dt
        = \int_0^\infty (-1)^n t^n f(t)e^{-st}dt \\
        &= (-1)^n \int_0^\infty t^n f(t)e^{-st}dt
        = (-1)^n \lap(t^n f(t))
    \end{align*}
\end{proof}

\begin{exercise}
    Calculate $\lap(t\cos(3t))$.
\end{exercise}
\begin{solution}
    \begin{align*}
        \lap(t\cos(3t))
        &= -\frac{d}{ds} \left(\frac{s}{s^2 + 9}\right)
        = \frac{2s^2 - s^2 - 9}{(s^2 + 9)^2}
        = \frac{s^2 - 9}{(s^2 + 9)^2}
    \end{align*}
\end{solution}

\subsection{Dirac Delta ``Function"}
\begin{definition}
    If $\delta_n(t) = \begin{cases}
        0 & . \\
        n & |t| \leq \frac{1}{2n}
    \end{cases}$
    then we define $\delta$ to be the Dirac-$\delta$ by the limit of the sequence of $\delta_n$'s so $\delta(t) = \lim_{n\to\infty} \delta_n(t)$.
\end{definition}

\begin{center}%
\begin{tikzpicture}
    \draw[<->] (-4, 0) -- (4, 0) node[right]{$t$};
    \draw[<->] (0, -1) -- (0, 3) node[above]{$\delta_n(t)$};
    
    \draw[thick] (-0.2, 2) -- (0.2, 2) node[above] {$n$};
    \draw[thick] (-2, 0) -- (-2, 0) node[below] {$-\frac{1}{2n}$};
    \draw[thick] (2, 0) -- (2, 0) node[below] {$\frac{1}{2n}$};

    \draw[line width = 1pt, draw = blue] (-4, 0) -- (-2, 0);
    \draw[line width = 1pt, draw = blue, dashed] (-2, 0) -- (-2, 2);
    \draw[fill = white, draw = blue] (-2, 0) circle (2pt);
    \filldraw[blue] (-2, 2) circle (2pt);
    \draw[line width = 1pt, draw = blue] (-2, 2) -- (2, 2);
    \draw[line width = 1pt, draw = blue, dashed] (2, 2) -- (2, 0);
    \draw[line width = 1pt, draw = blue] (2, 0) -- (4, 0);
    \draw[fill = white, draw = blue] (2, 0) circle (2pt);
    \filldraw[blue] (2, 2) circle (2pt);
\end{tikzpicture}
\end{center}

Notice that
\begin{align*}
    \int_{-\infty}^\infty \delta_n(t)dt
    &= \int_{-\frac{1}{2n}}^{\frac{1}{2n}} n dt
    = n\left(\frac{1}{n}\right)
    = 1
\end{align*}

\begin{theorem}
    If $f\colon [a, b] \to \real$ is continuous, $c \in [a, b]$, $\lambda_1 < c$ (or $\lambda_1 = -\infty$), and $\lambda_2 > c$ (or $\lambda_2 = \infty$) then
    \[
        \int_{\lambda_1}^{\lambda_2} f(t)\delta(t - c)dt = f(c)
    \]
\end{theorem}
\begin{proof}
    First consider that
    \begin{align*}
        \int_{\lambda_1}^{\lambda_2} f(t)\delta(t - c)dt
        &= \int_{\lambda_1}^{\lambda_2} \lim_{n\to\infty} f(t) \delta_n(t - c) dt \\
        &= \lim_{n\to\infty} \int_{\lambda_1}^{\lambda_2} f(t) \delta_n(t - c) dt \qquad \text{By Fatou's Lemma} \\
        &= \lim_{n\to\infty} \int_{c - \frac{1}{2n}}^{c + \frac{1}{2n}} nf(t)dt
        = \lim_{n\to\infty} n \int_{c - \frac{1}{2n}}^{c + \frac{1}{2n}} f(t)dt
    \end{align*}
    If we just evaluated the limit we would get an indeterminate form, so we want to bound the expression so we can use the squeeze theorem.

    Consider that for some $N$ that $c \pm \frac{1}{2n} \in [a, b]$ for $n \geq N$. Therefore, $f$ is continuous on $\left[c - \frac{1}{2n}, c + \frac{1}{2n}\right]$. Then, by the extreme value theorem, $f$ attains a minimum of $m_n = f(\alpha_n)$ and a maximum of $M_n = f(\beta_n)$ on $\left[c - \frac{1}{2n}, c + \frac{1}{2n}\right]$.

    Therefore $m_n \leq f(t) \leq M_n$ for $t \in \left[c - \frac{1}{2n}, c + \frac{1}{2n}\right]$. So we can say
    \begin{align*}
        &\int_{c - \frac{1}{2n}}^{c + \frac{1}{2n}} m_n dt \leq \int_{c - \frac{1}{2n}}^{c + \frac{1}{2n}} f(t)dt \leq \int_{c - \frac{1}{2n}}^{c + \frac{1}{2n}} M_n dt \\
        \iff& \frac{m_n}{n} \leq \int_{c - \frac{1}{2n}}^{c + \frac{1}{2n}} f(t)dt \leq \frac{M_n}{n} \\
        \iff& m_n \leq n\int_{c - \frac{1}{2n}}^{c + \frac{1}{2n}} f(t)dt \leq M_n
    \end{align*}

    Now we want to show that $\lim_{n\to\infty} m_n = \lim_{n\to\infty} M_n = f(c)$. \\ Consider that $c - \frac{1}{2n} \leq \alpha_n \leq c + \frac{1}{2n}$. Since $\lim_{n\to\infty} \left(c - \frac{1}{2n}\right) = \lim_{n\to\infty} \left(c + \frac{1}{2n}\right) = c$ then by the squeeze theorem $\lim_{n\to\infty} \alpha_n = c$. Therefore, $\lim_{n\to\infty} m_n = \lim_{n\to\infty} f(\alpha_n) = f\left(\lim_{n\to\infty} \alpha_n\right) = f(c)$. Without loss of generality, $\lim_{n\to\infty} M_n = f(c)$ as well. Therefore, by the squeeze theorem, $\lim_{n\to\infty} n\int_{c - \frac{1}{2n}}^{c + \frac{1}{2n}} f(t)dt = f(c)$.
\end{proof}

\begin{exercise}
    Find the value of $\int_0^\infty e^{-\pi t}\delta(t - \pi^2)dt$.
\end{exercise}
\begin{solution}
    \begin{align*}
        \int_0^\infty e^{-\pi t}\delta(t - \pi^2)dt
        &= e^{-\pi t}|_{t = \pi^2}
        = e^{-\pi^3}
    \end{align*}
\end{solution}

\begin{corollary}
    If $c > 0$ then $\lap(\delta(t - c)) = e^{-cs}$.
\end{corollary}
\begin{proof}
    \begin{align*}
        \lap(\delta(t - c))
        &= \int_0^\infty \delta(t - c)e^{-st}dt
        = e^{-st}|_{t = c}
        = e^{-cs}
    \end{align*}
\end{proof}

\begin{exercise}
    Solve $y'' + y = \delta(t - 2\pi)$ with $t = 0 \implies y = 0 \text{ and } y' = 1$.
\end{exercise}
\begin{solution}
    \begin{align*}
        &y'' + y = \delta(t - 2\pi) \\
        \iff& s^2 Y - sy|_{t = 0} - y'|_{t = 0} + Y = e^{-2\pi s} \\
        \iff& Y(s^2 + 1) = e^{-2\pi s} + 1 \\
        \iff& Y = \frac{e^{-2\pi s}}{s^2 + 1} + \frac{1}{s^2 + 1} \\
        \iff& y = \heav(t)\sin t|_{t = t - 2\pi} + \sin t = \heav(t - 2\pi)\sin t + \sin t
    \end{align*}
\end{solution}

% TODO: Add graph?

\end{document}