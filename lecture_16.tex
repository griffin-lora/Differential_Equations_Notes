\documentclass[notes]{subfiles}

\begin{document}

\setcounter{section}{16}
\section{Lecture (4/2/25)}

\subsection{Properties of the Laplace Transform}
\begin{lemma}
    The Laplace transform is a linear operator.
\end{lemma}
\begin{proof}
    \begin{align*}
        \lap (f + g)(s)
        &= \int_0^\infty (f + g)(t)e^{-st}dt
        = \int_0^\infty f(t)e^{-st}dt + \int_0^\infty g(t)e^{-st}dt
        = \lap f(s) + \lap g(s)
    \end{align*}
    Therefore $\lap (f + g) = \lap f + \lap g$.
    \begin{align*}
        \lap (cf)(s)
        &= \int_0^\infty (cf)(t)e^{-st}dt
        = c\int_0^\infty f(t)e^{-st}dt
        = c\lap f(s)
    \end{align*}
    Therefore $\lap (cf) = c\lap f$.
\end{proof}

\begin{theorem}
    $\lap (\cos(kt)) = \frac{s}{s^2 + k^2}$ and $\lap (\sin(kt)) = \frac{k}{s^2 + k^2}$ for $s > 0$.
\end{theorem}
\begin{proof}
    We are going to get both results for free if we notice that
    \[
        \lap (e^{ikt}) = \lap (\cos(kt)) + i\lap (\sin(kt))
    \]
    We are now going to evaluate the RHS so
    \begin{align*}
        \lap (e^{ikt})
        &= \int_0^\infty e^{ikt}e^{-st}dt
        = \lim_{b\to\infty} \int_0^b e^{-(s - ik)t}dt
        = \lim_{b\to\infty} \frac{-1}{s - ik}e^{-(s - ik)t} \Big|_{t = 0}^{t = b} \\
        &= -\frac{1}{s - ik} \lim_{b\to\infty} (e^{-(s - ik)b} - 1)
    \end{align*}
    Consider $\lim_{b\to\infty} |e^{-(s - ik)b}|$ so
    \begin{align*}
        \lim_{b\to\infty} |e^{-(s - ik)b}|
        &= \lim_{b\to\infty} |e^{-sb + ikb}|
        = \lim_{b\to\infty} |e^{-sb}e^{ikb}|
        = \lim_{b\to\infty} e^{-sb}
        = 0
    \end{align*}
    Since $\lim_{b\to\infty} |e^{-(s - ik)b}| = 0$ then $\lim_{b\to\infty} e^{-(s - ik)b} = 0$ so
    \begin{align*}
        \lap (e^{ikt})
        &= -\frac{1}{s - ik} \lim_{b\to\infty} (e^{-(s - ik)b} - 1)
        = \frac{1}{s - ik}
        = \frac{s + ik}{(s - ik)(s + ik)} \\
        &= \frac{s + ik}{s^2 + k^2}
        = \frac{s}{s^2 + k^2} + i\frac{k}{s^2 + k^2}
    \end{align*}
    Since $\lap (e^{ikt}) = \lap (\cos(kt)) + i\lap (\sin(kt))$ then
    \begin{align*}
        &\lap (\cos(kt)) + i\lap (\sin(kt)) = \frac{s}{s^2 + k^2} + i\frac{k}{s^2 + k^2} \\
        \iff& \lap (\cos(kt)) = \frac{s}{s^2 + k^2} \quad \text{and} \quad \lap (\sin(kt)) = \frac{k}{s^2 + k^2}
    \end{align*}
\end{proof}

We can now calculate Laplace transforms of linear combinations of polynomials, exponentials, $\sin$, and $\cos$.

\subsection{Inverse Laplace Transform}
The Laplace transform is useful since it can turn a linear IVP into an algebra problem. It can handle more complicated driver functions too, such as step functions or Dirac-$\delta$ ``functions".

The commutative diagram below describes the general process for using the inverse Laplace transform.

\begin{equation} \label{laplace_solution_diagram}
    \begin{tikzcd}
        {\text{Constant Coefficient Linear IVP}} &&& {\text{Equation involving $s$ and $Y$}} \\
        \\
        \\
        {\text{Solution function $t \mapsto y$}} &&& {\text{Rational/exponential function $s \mapsto Y$}}
        \arrow["{\lap }", from=1-1, to=1-4]
        \arrow["{\text{Earlier methods}}"', from=1-1, to=4-1]
        \arrow["{\text{Solve for function $s \mapsto Y$}}", from=1-4, to=4-4]
        \arrow["{\lap ^{-1}}"', from=4-4, to=4-1]
    \end{tikzcd}
\end{equation}

\begin{exercise}
    Find the Laplace transform of $f$, $F$ for $f$ as described by the graph
    \begin{center}%
    \begin{tikzpicture}
        \draw[<->] (-1, 0) -- (6, 0) node[right]{$t$};
        \draw[<->] (0, -1) -- (0, 3) node[above]{$f(t)$};
        
        \draw[thick] (-0.2, 2) -- (0.2, 2) node[right] {$c$};
        \draw[thick] (2, 0) -- (2, 0) node[below] {$a$};
        \draw[thick] (4, 0) -- (4, 0) node[below] {$b$};

        \draw[line width = 1pt, draw = blue] (0, 0) -- (2, 0);
        \draw[line width = 1pt, draw = blue, dashed] (2, 0) -- (2, 2);
        \draw[fill = white, draw = blue] (2, 0) circle (2pt);
        \filldraw[blue] (2, 2) circle (2pt);
        \draw[line width = 1pt, draw = blue] (2, 2) -- (4, 2);
        \draw[line width = 1pt, draw = blue, dashed] (4, 2) -- (4, 0);
        \draw[fill = white, draw = blue] (4, 2) circle (2pt);
        \filldraw[blue] (4, 0) circle (2pt);
        \draw[line width = 1pt, draw = blue] (4, 0) -- (6, 0);
    \end{tikzpicture}
    \end{center}
\end{exercise}

\begin{solution}
    From the graph above, we can say that
    \[
        f(t) = \begin{cases}
            0 & 0 \leq t < a \\
            c & a \leq t < b \\
            0 & t \geq b
        \end{cases}
    \]
    Therefore we can consider the Laplace transform so
    \begin{align*}
        F(s)
        &= \int_0^\infty f(t)e^{-st}dt
        = \int_0^a 0 + \int_a^b ce^{-st}dt + \int_b^\infty 0
        = \int_a^b ce^{-st}dt
    \end{align*}
    If $s = 0$ then
    \begin{align*}
        F(s)
        &= \int_a^b ce^{-st}dt
        = c\int_a^b dt
        = c(b - a)
    \end{align*}
    If $s \neq 0$ then
    \begin{align*}
        F(s)
        &= \int_a^b ce^{-st}dt
        = -\frac{c}{s}e^{-st}\Big|_a^b
        = -\frac{c}{s}(e^{-bs} - e^{-as})
    \end{align*}
    Therefore
    \[
        F(s) = \begin{cases}
            c(b - a) & s = 0 \\
            -\frac{c}{s}(e^{-bs} - e^{-as}) & s \neq 0
        \end{cases}
    \]
\end{solution}

Technically the Laplace transform is not injective. But we are in an engineering class so who cares.
\begin{definition}[Inverse Laplace Transform]
    If $\lap f = F$ then the \textsl{inverse Laplace transform} $\lap ^{-1}$ is such that $\lap ^{-1}F = f$.
\end{definition}

\begin{lemma}
    The inverse Laplace transform is a linear map.
\end{lemma}
\begin{proof}
    Since the Laplace transform is a linear map, then linear algebra tells us that the inverse of a linear map is also a linear map. Therefore, the inverse Laplace transform is a linear map.
\end{proof}

Recall that $\lap (t^n) = \frac{n!}{s^{n + 1}}$, $\lap (e^{at}) = \frac{1}{s - a}$, $\lap (\cos(kt)) = \frac{s}{s^2 + k^2}$, and $\lap (\sin(kt)) = \frac{k}{s^2 + k^2}$. Therefore, if we want to calculate the inverse Laplace transform, we get the function in terms of the RHS of the equations above.

\begin{exercise}
    Calculate the inverse Laplace transform $f$ of $F(s) = \frac{1}{s^6} + \frac{1}{s^2 + 7}$.
\end{exercise}
\begin{solution}
    \begin{align*}
        f(t)
        &= \lap ^{-1}\left(\frac{1}{s^6} + \frac{1}{s^2 + 7}\right)
        = \lap ^{-1}\left(\frac{1}{s^6}\right) + \lap ^{-1}\left(\frac{1}{s^2 + 7}\right) \\
        &= \frac{1}{5!}\lap ^{-1}\left(\frac{5!}{s^6}\right) + \frac{1}{\sqrt{7}}\lap ^{-1}\left(\frac{\sqrt{7}}{s^2 + 7}\right)
        = \frac{1}{5!}t^5 + \frac{1}{\sqrt{7}}\sin(\sqrt{7}t)
    \end{align*}
\end{solution}

\end{document}