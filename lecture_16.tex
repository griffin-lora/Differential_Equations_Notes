\documentclass[notes]{subfiles}
\externaldocument{lecture_11}

\begin{document}

\setcounter{section}{16}
\section{Lecture (4/2/25)}

\subsection{Properties of the Laplace Transform}
\begin{lemma}
    The Laplace transform is a linear operator.
\end{lemma}
\begin{proof}
    \begin{align*}
        \mathcal{L}(f + g)(s)
        &= \int_0^\infty (f + g)(t)e^{-st}dt
        = \int_0^\infty f(t)e^{-st}dt + \int_0^\infty g(t)e^{-st}dt
        = \mathcal{L}f(s) + \mathcal{L}g(s)
    \end{align*}
    Therefore $\mathcal{L}(f + g) = \mathcal{L}f + \mathcal{L}g$.
    \begin{align*}
        \mathcal{L}(cf)(s)
        &= \int_0^\infty (cf)(t)e^{-st}dt
        = c\int_0^\infty f(t)e^{-st}dt
        = c\mathcal{L}f(s)
    \end{align*}
    Therefore $\mathcal{L}(cf) = c\mathcal{L}f$.
\end{proof}

\begin{theorem}
    $\mathcal{L}(\cos(kt)) = \frac{s}{s^2 + k^2}$ and $\mathcal{L}(\sin(kt)) = \frac{k}{s^2 + k^2}$ for $s > 0$.
\end{theorem}
\begin{proof}
    We are going to get both results for free if we notice that
    \[
        \mathcal{L}(e^{ikt}) = \mathcal{L}(\cos(kt)) + i\mathcal{L}(\sin(kt))
    \]
    We are now going to evaluate the RHS so
    \begin{align*}
        \mathcal{L}(e^{ikt})
        &= \int_0^\infty e^{ikt}e^{-st}dt
        = \lim_{b\to\infty} \int_0^b e^{-(s - ik)t}dt
        = \lim_{b\to\infty} \frac{-1}{s - ik}e^{-(s - ik)t} \Big|_{t = 0}^{t = b} \\
        &= -\frac{1}{s - ik} \lim_{b\to\infty} (e^{-(s - ik)b} - 1)
    \end{align*}
    Consider $\lim_{b\to\infty} |e^{-(s - ik)b}|$ so
    \begin{align*}
        \lim_{b\to\infty} |e^{-(s - ik)b}|
        &= \lim_{b\to\infty} |e^{-sb + ikb}|
        = \lim_{b\to\infty} |e^{-sb}e^{ikb}|
        = \lim_{b\to\infty} e^{-sb}
        = 0
    \end{align*}
    Since $\lim_{b\to\infty} |e^{-(s - ik)b}| = 0$ then $\lim_{b\to\infty} e^{-(s - ik)b} = 0$ so
    \begin{align*}
        \mathcal{L}(e^{ikt})
        &= -\frac{1}{s - ik} \lim_{b\to\infty} (e^{-(s - ik)b} - 1)
        = \frac{1}{s - ik}
        = \frac{s + ik}{(s - ik)(s + ik)} \\
        &= \frac{s + ik}{s^2 + k^2}
        = \frac{s}{s^2 + k^2} + i\frac{k}{s^2 + k^2}
    \end{align*}
    Since $\mathcal{L}(e^{ikt}) = \mathcal{L}(\cos(kt)) + i\mathcal{L}(\sin(kt))$ then
    \begin{align*}
        &\mathcal{L}(\cos(kt)) + i\mathcal{L}(\sin(kt)) = \frac{s}{s^2 + k^2} + i\frac{k}{s^2 + k^2} \\
        \iff& \mathcal{L}(\cos(kt)) = \frac{s}{s^2 + k^2} \quad \text{and} \quad \mathcal{L}(\sin(kt)) = \frac{k}{s^2 + k^2}
    \end{align*}
\end{proof}

We can now calculate Laplace transforms of linear combinations of polynomials, exponentials, $\sin$, and $\cos$.

\subsection{Inverse Laplace Transform}
The Laplace transform is useful since it can turn a linear IVP into an algebra problem. It can handle more complicated driver functions too, such as step functions or Dirac-$\delta$ ``functions".

The commutative diagram below describes the general process for using the inverse Laplace transform.

\[
    \begin{tikzcd}
        {\text{Constant Coefficient Linear IVP}} &&& {\text{Equation involving $s$ and $Y$}} \\
        \\
        \\
        {\text{Solution function $t \mapsto y$}} &&& {\text{Rational/exponential function $s \mapsto Y$}}
        \arrow["{\mathcal{L}}", from=1-1, to=1-4]
        \arrow["{\text{Earlier methods}}"', from=1-1, to=4-1]
        \arrow["{\text{Solve for function $s \mapsto Y$}}", from=1-4, to=4-4]
        \arrow["{\mathcal{L}^{-1}}"', from=4-4, to=4-1]
    \end{tikzcd}
\]

\begin{exercise}
    Find the Laplace transform of $f$, $F$ for $f$ as described by the graph
    \begin{center}%
    \begin{tikzpicture}
        \draw[<->] (-1, 0) -- (6, 0) node[right]{$t$};
        \draw[<->] (0, -1) -- (0, 3) node[above]{$f(t)$};
        
        \draw[thick] (-0.2, 2) -- (0.2, 2) node[right] {$c$};
        \draw[thick] (2, 0) -- (2, 0) node[below] {$a$};
        \draw[thick] (4, 0) -- (4, 0) node[below] {$b$};

        \draw[line width = 1pt, draw = blue] (0, 0) -- (2, 0);
        \draw[line width = 1pt, draw = blue, dashed] (2, 0) -- (2, 2);
        \draw[fill = white, draw = blue] (2, 0) circle (2pt);
        \filldraw[blue] (2, 2) circle (2pt);
        \draw[line width = 1pt, draw = blue] (2, 2) -- (4, 2);
        \draw[line width = 1pt, draw = blue, dashed] (4, 2) -- (4, 0);
        \draw[fill = white, draw = blue] (4, 2) circle (2pt);
        \filldraw[blue] (4, 0) circle (2pt);
        \draw[line width = 1pt, draw = blue] (4, 0) -- (6, 0);
    \end{tikzpicture}
    \end{center}
\end{exercise}

\end{document}