\documentclass[notes]{subfiles}
\externaldocument{lecture_11}

\begin{document}

\setcounter{section}{15}
\section{Lecture (3/31/25)}

\subsection{Laplace Transform}
The Laplace transorm is a special type of operator which does not use the same input variable as output variable. Instead of being able to consider the transform at each point of a function's domain, we consider the entire function as transformed, with no direct connection between the domains.

\begin{definition}[Integral Transform]
    Given a function $t \mapsto f(t)$, we define an \textit{integral transform} of $f$ as $s \mapsto F(s)$ such that
    \[
        F(s) = \int_a^b f(t)K(s, t)dt
    \]
    where $a, b$ are constant and $K$ is the \textit{kernel function} of the integral transform.
\end{definition}

\begin{example}
    The Fourier transform is an integral transform such that $a = -\infty$, $b = \infty$ and $K(s, t) = e^{2\pi i st}$.
\end{example}

\begin{definition}[Exponential Type]
    A univariate function $f$ is of \textit{exponential type} iff $|f(t)| \leq Me^{kt}$ where $M, k \in (0, \infty)$.
\end{definition}

\begin{example}[Exponential type functions]
    ~\par
    \begin{enumerate}[label = (\arabic*)]
        \item All polynomials are of exponential type.
        \item $\sin$ and $\cos$ are of exponential type.
        \item $t \mapsto ce^{at}$ for constants $c, a$
        \item Any bounded function is of exponential type.
    \end{enumerate}
\end{example}

\begin{example}[Non-exponential type functions]
    ~\par
    \begin{enumerate}[label = (\arabic*)]
        \item $t \mapsto \frac{1}{t}$ is not of exponential type since $\lim_{t \to 0^+} \frac{1}{t}$ is an infinite limit.
        \item $t \mapsto e^{t^2}$
        \item $\tan$
        \item $t \mapsto t^t$
    \end{enumerate}
\end{example}

\begin{definition}[Laplace Transform]
    Given a function $t \mapsto f(t)$ of exponential type, then the \textit{Laplace transform} $s \mapsto (\mathcal{L}f)(s)$ is the integral transform defined by
    \[
        (\mathcal{L}f)(s) = \int_0^\infty f(t)e^{-st}dt
    \]
    Note that we usually notate $F = \mathcal{L}f$, $G = \mathcal{L}g$, and so on.
\end{definition}

\begin{theorem}
    $\dom \mathcal{L}f = I$ where $I$ is an interval in $\mathbb{R}$.
\end{theorem}

We can view the Laplace transform as a continuous version of a generating function. With a generating function we get $a_n \mapsto \sum_{n = 0}^\infty a_n x^n$. In comparison, a Laplace transform gives us $f(t) \mapsto \int_0^\infty f(t)e^{-st}dt$.

\begin{exercise} \label{laplace_transform_base_case}
    Find the Laplace transform of $f$ as $f(t) = 1$.
\end{exercise}
\begin{solution}
    \begin{align*}
        F(s)
        &= \int_0^\infty f(t)e^{-st}dt
        = 
    \end{align*}
\end{solution}

\end{document}