\documentclass[notes]{subfiles}

\begin{document}

\setcounter{section}{19}
\section{Lecture (4/16/25)}

\subsection{Convolution Theorem}
\begin{definition}
    \textsl{Convolution} is a binary operation $\ast$ that takes real-valued functions $f$ and $g$ such that
    \[
        (f\ast g)(t) = \int_0^t f(u)g(t - u)du
    \]
\end{definition}

\begin{lemma}
    Convolution is commutative.
\end{lemma}
\begin{proof}
    Consider functions $f$ and $g$ so
    \begin{align*}
        (f\ast g)(t)
        &= \int_0^t f(u)g(t - u)du
    \end{align*}
    Now let $v = t - u \implies dv = -du \iff u = t - v$ so
    \begin{align*}
        (f\ast g)(t)
        &= \int_0^t f(u)g(t - u)du
        = -\int_t^0 f(t - v)g(v)dv \\
        &= \int_0^t g(v)f(t - v)dv
        = \int_0^t g(u)f(t - u)dt \\
        &= (g \ast f)(t)
    \end{align*}
\end{proof}

\begin{theorem}[Convolution Theorem]
    If $f$ and $g$ are functions then
    \[
        \lap(f \ast g) = (\lap f)(\lap g)
    \]
\end{theorem}

\begin{exercise}
    Calculate $\lap^{-1}\left( \frac{1}{(s^2 + 9)^2} \right)$.
\end{exercise}
\begin{solution}
    \begin{align*}
        \lap^{-1}\left( \frac{1}{(s^2 + 9)^2} \right)
        &= \lap^{-1}\left( \frac{1}{s^2 + 9} \right) \ast \lap^{-1}\left( \frac{1}{s^2 + 9} \right)
        = \left(\frac{1}{3}\sin(3t)\right)\ast\left(\frac{1}{3}\sin(3t)\right) \\
        &= \frac{1}{9} \int_0^t \sin(3u)\sin(3(t - u))du \\
        &= \frac{1}{18} \int_0^t ( \cos(3u - 3(t - u)) - \cos(3u + 3(t - u)) )du \\
        &= \frac{1}{18} \int_0^t ( \cos(-3t + 6u) - \cos(3t) )du \\
        &= \frac{1}{18} \left( \frac{1}{6}\sin(-3t + 6u) - u\cos(3t) \right)\Big|_{u = 0}^{u = t} \\
        &= \frac{1}{18} \left( \frac{1}{6}\sin(3t) - t\cos(3t) - \frac{1}{6}\sin(-3t) \right) \\
        &= \frac{1}{18} \left( \frac{1}{3}\sin(3t) - t\cos(3t) \right)
    \end{align*}
\end{solution}

Another example of convolution is multiplying series together.
\[
    \left( \sum_{n = 1}^\infty a_n \right)\left( \sum_{n = 1}^\infty b_n \right) = \sum_{n = 1}^\infty \sum_{k = 1}^n a_k b_{n - k}
\]

\subsection{Brief Review of Power Series}
\begin{definition}
    A series is a \textsl{power series centered at} $a$ iff it is of the form
    \[
        \sum_{n = p}^\infty c_n(x - a)^n
    \]
    where sequence $(c_n)_{n = p}^\infty$ is called the \textsl{coefficient sequence}. 
\end{definition}

\end{document}