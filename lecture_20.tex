\documentclass[notes]{subfiles}

\begin{document}

\setcounter{section}{19}
\section{Lecture (4/16/25)}

\subsection{Convolution Theorem}
\begin{definition}
    \textsl{Convolution} is a binary operation $\ast$ that takes real-valued functions $f$ and $g$ such that
    \[
        (f\ast g)(t) = \int_0^t f(u)g(t - u)du
    \]
\end{definition}

\begin{lemma}
    Convolution is commutative.
\end{lemma}
\begin{proof}
    Consider functions $f$ and $g$ so
    \begin{align*}
        (f\ast g)(t)
        &= \int_0^t f(u)g(t - u)du
    \end{align*}
    Now let $v = t - u \implies dv = -du \iff u = t - v$ so
    \begin{align*}
        (f\ast g)(t)
        &= \int_0^t f(u)g(t - u)du
        = -\int_t^0 f(t - v)g(v)dv \\
        &= \int_0^t g(v)f(t - v)dv
        = \int_0^t g(u)f(t - u)dt \\
        &= (g \ast f)(t)
    \end{align*}
\end{proof}

\begin{theorem}[Convolution Theorem]
    If $f$ and $g$ are functions then
    \[
        \lap(f \ast g) = (\lap f)(\lap g)
    \]
\end{theorem}

\begin{exercise}
    Calculate $\lap^{-1}\left( \frac{1}{(s^2 + 9)^2} \right)$.
\end{exercise}
\begin{solution}
    \begin{align*}
        \lap^{-1}\left( \frac{1}{(s^2 + 9)^2} \right)
        &= \lap^{-1}\left( \frac{1}{s^2 + 9} \right) \ast \lap^{-1}\left( \frac{1}{s^2 + 9} \right)
        = \left(\frac{1}{3}\sin(3t)\right)\ast\left(\frac{1}{3}\sin(3t)\right) \\
        &= \frac{1}{9} \int_0^t \sin(3u)\sin(3(t - u))du \\
        &= \frac{1}{18} \int_0^t ( \cos(3u - 3(t - u)) - \cos(3u + 3(t - u)) )du \\
        &= \frac{1}{18} \int_0^t ( \cos(-3t + 6u) - \cos(3t) )du \\
        &= \frac{1}{18} \left( \frac{1}{6}\sin(-3t + 6u) - u\cos(3t) \right)\Big|_{u = 0}^{u = t} \\
        &= \frac{1}{18} \left( \frac{1}{6}\sin(3t) - t\cos(3t) - \frac{1}{6}\sin(-3t) \right) \\
        &= \frac{1}{18} \left( \frac{1}{3}\sin(3t) - t\cos(3t) \right)
    \end{align*}
\end{solution}

Another example of convolution is multiplying series together.
\[
    \left( \sum_{n = 1}^\infty a_n \right)\left( \sum_{n = 1}^\infty b_n \right) = \sum_{n = 1}^\infty \sum_{k = 1}^n a_k b_{n - k}
\]

\subsection{Brief Review of Power Series}
% \begin{theorem}[Series Divergence Theorem] \label{series_divg_thm}
%     Given a series $\sum_{n = p}^\infty a_n$ and $\lim_{n\to\infty} a_n \neq 0$ then $\sum_{n = p}^\infty a_n$ diverges.
% \end{theorem}

\begin{theorem}[Ratio Convergence Theorem] \label{ratio_cnvg_thm}
    Given a series $\sum_{n = p}^\infty a_n$ and $L = \lim_{n\to\infty} \left| \frac{a_{n + 1}}{a_n} \right|$ then
    \begin{align*}
        L < 1 &\implies \text{$\sum_{n = p}^\infty a_n$ converges absolutely} \\
        L > 1 &\implies \text{$\sum_{n = p}^\infty a_n$ diverges}
    \end{align*}
\end{theorem}

% \begin{theorem}[Root Convergence Theorem]
%     Given a series $\sum_{n = p}^\infty a_n$ and $L = \lim_{n\to\infty} \sqrt[n]{|a_n|}$ then
%     \begin{align*}
%         L < 1 &\implies \text{$\sum_{n = p}^\infty a_n$ converges absolutely} \\
%         L > 1 &\implies \text{$\sum_{n = p}^\infty a_n$ diverges}
%     \end{align*}
% \end{theorem}

\begin{definition}[Power Series]
    A series is a \textsl{power series centered at} $a$ iff it is of the form
    \[
        \sum_{n = p}^\infty c_n(x - a)^n
    \]
    where sequence $(c_n)_{n = p}^\infty$ is called the \textsl{coefficient sequence}. 
\end{definition}

\begin{definition}[Radius of Convergence]
    $R$ is the \textsl{radius of convergence} for a power series $\Sigma$ centered at $a$ iff
    \begin{align*}
        |x - a| < R &\implies \text{$\Sigma$ converges absolutely} \\
        |x - a| > R &\implies \text{$\Sigma$ diverges}
    \end{align*}
\end{definition}

The boundary conditions of $|x - a| = R$ are not as easily well defined.

\begin{theorem}
    Given a power series, then it has a unique radius of convergence $R \geq 0$.
\end{theorem}

\begin{definition}
    $I$ is the \textsl{interval of convergence} for a power series $\Sigma$ iff $\Sigma$ converges on $I$ and diverges outside of $I$.
\end{definition}

\begin{lemma}
    If $I$ is an interval of convergence for $\Sigma$ centered at $a$ with a radius of convergence $R$ then it is of one of the following forms
    \begin{enumerate}[label = (\arabic*)]
        \item $I = (a - R, a + R)$
        \item $I = [a - R, a + R)$
        \item $I = (a - R, a + R]$
        \item $I = [a - R, a + R]$
    \end{enumerate}
\end{lemma}

\begin{exercise}
    Find the interval of convergence for
    \[
        \sum_{n = 1}^\infty (-1)^n \frac{(2n - 1)!!}{n!}(x - 3)^n
    \]
\end{exercise}
\begin{solution}
    Consider
    \begin{align*}
        L
        &= \lim_{n\to\infty} \left| \frac{(2n + 1)!! (x - 3)^{n + 1} n!}{(n + 1)! (2n - 1)!! (x - 3)^n} \right| \\
        &= \lim_{n\to\infty} \left| \frac{2n + 1}{n + 1} (x - 3) \right|
        = |x - 3| \lim_{n\to\infty} \left| \frac{2 + \frac{1}{n}}{1 + \frac{1}{n}} \right| \\
        &= 2|x - 3|
    \end{align*}
    Now consider
    \begin{align*}
        L < 1
        &\iff 2|x - 3| < 1
        \iff |x - 3| < \frac{1}{2}
    \end{align*}
    Therefore by the \cref{ratio_cnvg_thm}, $R = \frac{1}{2}$. Therefore, the boundary values are $\frac{5}{2}$ and $\frac{3}{2}$. At $x = \frac{5}{2}$ we get
    \begin{align*}
        \sum_{n = 1}^\infty (-1)^n \frac{(2n - 1)!!}{n!}(x - 3)^n
        &= \sum_{n = 1}^\infty (-1)^n \frac{(2n - 1)!!}{n!}\left(-\frac{1}{2}\right)^n \\
        &= \sum_{n = 1}^\infty \frac{(2n - 1)!!}{n! 2^n}
        = \sum_{n = 1}^\infty \frac{(2n - 1)!!}{(2n)!!}
    \end{align*}
    By fiddling around on Desmos for an hour, we can find out this problem is too annoying so it's probably divergent. For the case where $x = \frac{3}{2}$ similar reasoning gives us that its probably convergent. Therefore, the interval of convergence is $\left[ \frac{3}{2}, \frac{5}{2} \right)$.
\end{solution}

Important power series:
\begin{enumerate}[label = (\arabic*)]
    \item $e^x = \sum_{n = 0}^\infty \frac{x^n}{n!}$, $R = \infty$
    \item $\frac{1}{1 - x} = \sum_{n = 0}^\infty x^n$, $R = 1$
    \item $\sin x = \sum_{n = 0}^\infty (-1)^n \frac{x^{2n + 1}}{(2n + 1)!}$, $R = \infty$
    \item $\cos x = \sum_{n = 0}^\infty (-1)^n \frac{x^{2n}}{(2n)!}$, $R = \infty$
    \item $\ln(1 + x) = \sum_{n = 0}^\infty (-1)^n \frac{x^{n + 1}}{n + 1}$, $R = 1$
    \item $\atan x = \sum_{n = 0}^\infty (-1)^n \frac{x^{2n + 1}}{2n + 1}$, $R = 1$
\end{enumerate}

\begin{definition}[Real Analytic]
    A real-valued function $f$ is \textsl{analytic} at $a$ iff $f$ has a power series representation centered at $a$ such that $R > 0$.
\end{definition}

For this class, functions that are defined at a point are going to be analytic at that point.

\begin{theorem}
    If a power series $\sum_{n = 0}^\infty a_n(x - a)^n$ converges on an open interval $I$ then the following is true for all $x \in I$
    \[
        \frac{d}{dx} \sum_{n = 0}^\infty a_n(x - a)^n = \sum_{n = 1}^\infty nc_n(x - a)^{n - 1}
    \]
\end{theorem}

\subsection{Solving Linear ODEs with Power Series}

\begin{definition}[Ordinary Point]
    Given an order two linear ODE in the form $y'' + P(x)y' + Q(x)y = f(x)$ then $x_0$ is an \textsl{ordinary point} for the ODE iff $P$ and $Q$ are analytic at $x_0$.
\end{definition}

\begin{definition}[Singular Point]
    $x_0$ is a singular point for an ODE iff it is not an ordinary point for an ODE.
\end{definition}

To solve a linear ODE about an ordinary point we assume an ordinary solution so
\[
    y = \sum_{n = 0}^\infty c_n(x - x_0)^n \implies y' = \sum_{n = 1}^\infty nc_n(x - x_0)^{n - 1} \implies y'' = \sum_{n = 2}^\infty n(n - 1)c_n(x - x_0)^{n - 2}
\]
Then we build the ODE, and merge the series to construct a recurrence relation for the $c_n$'s.

\begin{exercise}
    Solve $y'' - xy = 0$ about $x_0 = 0$ with power series by expanding the first few terms.
\end{exercise}
\begin{solution}
    Building the ODE we get 
    \begin{align*}
        0 &= \sum_{n = 2}^\infty n(n - 1)c_nx^{n - 2} - x\sum_{n = 0}^\infty c_nx ^n \\
        &= \sum_{n = 0}^\infty (n + 2)(n + 1)c_{n + 2}x^n - \sum_{n = 0}^\infty c_nx^{n + 1} \\
        &= 2c_2 + \sum_{n = 1}^\infty (n + 2)(n + 1)c_{n + 2}x^n - \sum_{n = 1}^\infty c_{n - 1}x^n \\
        &= 2c_2 + \sum_{n = 1}^\infty ((n + 2)(n + 1)c_{n + 2} - c_{n - 1})x^n
    \end{align*}
    Therefore $c_2 = 0$ and since we are analytic then for $n \geq 1$
    \begin{align*}
        &(n + 2)(n + 1)c_{n + 2} - c_{n - 1} = 0
        \iff (n + 2)(n + 1)c_{n + 2} = c_{n - 1} \\
        \iff& c_{n + 2} = \frac{c_{n - 1}}{(n + 2)(n + 1)}
    \end{align*}
    Notice that $c_0$ and $c_1$ are undetermined. This is because these represent the coordinates for our basis solution (as this is a homogeneous ODE).

    If we expand out a few terms we get
    \begin{align*}
        c_3 &= \frac{c_0}{(3)(2)} \\
        c_4 &= \frac{c_1}{(4)(3)} \\
        c_5 &= \frac{c_2}{(5)(4)} = 0 \\
        c_6 &= \frac{c_3}{(6)(5)} = \frac{c_0}{(3)(2)(5)(6)} \\
        c_7 &= \frac{c_4}{(7)(6)} = \frac{c_1}{(3)(4)(6)(7)}
    \end{align*}
    We can now see that $c_{4n + 1} = 0$ for $n \geq 1$ so
    \begin{align*}
        y
        &= c_0 + c_1x + \frac{c_0}{(3)(2)}x^3 + \frac{c_1}{(4)(3)}x^4 + \frac{c_0}{(3)(2)(5)(6)}x^6 + \frac{c_1}{(3)(4)(6)(7)}x^7 + \ldots \\
        &= c_0\left(1 + \frac{1}{(3)(2)}x^3 + \frac{1}{(3)(2)(5)(6)}x^6 + \ldots \right) + c_1\left(x + \frac{1}{(4)(3)}x^4 + \frac{1}{(3)(4)(6)(7)}x^7 + \ldots \right)
    \end{align*}
\end{solution}

\end{document}