\documentclass[notes]{subfiles}

\begin{document}

\setcounter{section}{10}
\section{Lecture (2/24/25)}
\subsection{Rational Roots Theorem and Synthetic Division}
Consider the polynomial $P(x) = 4(x - 3)(x + 1)(x - 5)$. Notice that
\begin{align*}
    P(x)
    &= 4x^3 + \ldots + 4(-3)(1)(-5)
    = 4x^3 + \ldots + 60
    = a_3x^3 + \ldots + a_0
\end{align*}
Notice that the leading factor ($4$) of the polynomial divides $a_3$ and $a_0$. Also notice that the roots of the polynomial also divide $a_0$.

Now consider
\begin{align*}
    P(x)
    &= 4(2x - 1)(x + 3)(5x + 2)
    = 4(2)(1)(5)x^3 + \ldots + 4(-1)(3)(2) \\
    &= a_3x^3 + \ldots + a_0
\end{align*}
In this polynomial, in addition to the leading factor of $4$, the denominators of the roots divide $a_3$, while the numerators of the roots divide $a_0$.

\begin{theorem}[Rational Roots Theorem] \label{rational_roots_theorem}
    Given a polynomial in the form
    \[
        P(x) = \sum_{i = 0}^n a_i x^i
    \]
    where $[a_i]_{i = 0}^n \in \mathbb{Z}$, $a_0, a_n \neq 0$, then all rational roots are of the form
    \[
        x = \pm \frac{p}{q}
    \]
    where $p \mid a_0$ and $q \mid a_n$.
\end{theorem}
% \begin{proof}
%     By the fundamental theorem of algebra, there exists $a, [\lambda_i]_{i = 1}^n \in \real$ such that
%     \[
%         P(x) = a\prod_{i = 1}^n (x - \lambda_i)
%     \]
%     Consider just the rational roots $[r_i]_{i = 1}^m \in \{ \lambda_i \}_{i = 1}^n \cap \mathbb{Q}$. Therefore for $i = 1$ to $m$
%     \[
%         r_i = \frac{p_i}{q_i} \quad \text{where $p_i, q_i \in \mathbb{Z}$}
%     \]
%     Now let the remaining factors be $\lambda$ Therefore
%     \begin{align*}
%         P(x)
%         &= a\prod_{i = 1}^n (x - \lambda_i)
%         = a\lambda \prod_{i = 1}^m (x - r_i) \\
%         &= a\lambda \prod_{i = 1}^m \left( x - \frac{p_i}{q_i} \right)
%         = a\lambda \prod_{i = 1}^m \frac{1}{q_i} ( q_i x - p_i )
%     \end{align*}
% \end{proof}

\begin{procedure}[Factoring polynomials with integer coefficients]
    ~

    \begin{enumerate}[label = (\arabic*)]
        \item Use \cref{rational_roots_theorem} to find all possible factors for the polynomial.
        \item Divide the polynomial by each factor, a remainder of $0$ means that a factor has been found.
        \item Repeat until all factors of the polynomial are found.
    \end{enumerate}
\end{procedure}

\begin{exercise}
    Factor $2x^4 + 5x^3 + 5x^2 + 20x - 12$.
\end{exercise}
\begin{solution}
    Consider that $1$ and $2$ divide $2$, and $1$, $2$, $3$, $4$, $6$, and $12$ divide $12$. By \cref{rational_roots_theorem} we have possible roots of
    \[
        \pm \left( 2, 3, 4, 6, 12, \frac{1}{2}, \frac{3}{2} \right)
    \]
    Checking with synthetic division we find that $\frac{1}{2}$ is a root therefore
    % \[
    %     \begin{array}{cc}
    %         \begin{array}{c} \\ -2 \\ \\ \end{array}
    %         &
    %         \begin{array}{|rrrrr}  
    %             2 & 5 & 5 & 20 & -12 \\
    %             & -4 & -2 & -6 & -28 \\
    %             \hline 
    %             2 & 1 & 3 & 14 & -40
    %         \end{array}
    %     \end{array}
    % \]
    % \[
    %     \begin{array}{cc}
    %         \begin{array}{c} \\ 2 \\ \\ \end{array}
    %         &
    %         \begin{array}{|rrrrr}  
    %             2 & 5 & 5 & 20 & -12 \\
    %             & -4 & -2 & -6 & -28 \\
    %             \hline 
    %             2 & 1 & 3 & 14 & -40
    %         \end{array}
    %     \end{array}
    % \]
    \begin{align*}
        2x^4 + 5x^3 + 5x^2 + 20x - 12
        &= \left( x - \frac{1}{2} \right)(2x^3 + 6x^2 + 8x + 24) \\
        &= 2\left( x - \frac{1}{2} \right)(x^3 + 3x^2 + 4x + 12)
    \end{align*}
    Now consider that $1$ divides $1$, and $1$, $2$, $3$, $4$, $6$, and $12$ divide $12$. Therefore we have possible roots of
    \[
        \pm \left( 2, 3, 4, 6, 12 \right)
    \]
    By synthetic division we find that $-3$ is a root therefore
    \begin{align*}
        x^3 + 3x^2 + 4x + 12
        &= (x + 3)(x^2 + 4)
    \end{align*}
    Since $x^2 + 4$ is an irreducible quadratic factor, we are done factoring.
    
    Therefore the polynomial can be factored as
    \[
        2x^4 + 5x^3 + 5x^2 + 20x - 12 = 2\left( x - \frac{1}{2} \right)(x + 3)(x^2 + 4)
    \]
\end{solution}

\subsection{Constant Coefficient Linear Homogeneous ODEs}
The main idea here is to change constant coefficient linear homogeneous ODEs into their characteristic polynomial and to then solve the polynomial. We will then obtain solutions in terms of powers of $x$ and exponentials.

By linear algebra, functions in the form of $f(x) = x^p e^{mx}$ generate the null space of linear differential operators. Therefore, we let $y = e^{mx}$ be a solution and consider the differential equation in terms of that solution.

\begin{exercise}
    Solve $y'' - 16y = 0$.
\end{exercise}
\begin{solution}
    $y = e^{mx} \iff y'' = m^2 e^{mx}$.
    \begin{align*}
        m^2 e^{mx} - 16e^{mx} = 0
        &\iff m^2 - 16 = 0
    \end{align*}
    For the characteristic polynomial then
    \begin{align*}
        0
        &= m^2 - 16
        = (m - 4)(m + 4)
    \end{align*}
    So $m = \pm 4$.
    Therefore $y = c_1 e^{-4x} + c_2 e^{4x}$ solves the ODE.
\end{solution}

If we do not get all basis solutions then we can then use reduction of order to find other solutions.

\begin{exercise}
    Solve $y'' - 2y' + y = 0$.
\end{exercise}
\begin{solution}
    $y = e^{mx} \iff y' = me^{mx} \iff y'' = m^2e^{mx}$.
    \begin{align*}
        m^2e^{mx} - 2me^{mx} + e^{mx} = 0
        &\iff m^2 - 2m + 1 = 0
    \end{align*}
    For the characteristic polynomial then
    \begin{align*}
        0
        &= m^2 - 2m + 1
        = (m - 1)^2
    \end{align*}
    Therefore $m = 1$ so $y_1 = e^x$. Using reduction of order then
    \begin{align*}
        y_2
        &= y_1 \int \frac{e^{-\int P(x)dx}}{y_1^2}dx
        = e^x \int \frac{e^{\int 2dx}}{e^{2x}}dx \\
        &= e^x \int \frac{e^{2x}}{e^{2x}}dx
        = e^x \int dx
        = xe^x 
    \end{align*}
    Therefore, $y = c_1e^x + c_2xe^x$ solves the ODE.
\end{solution}

There are three cases when solving constant coefficient linear homogeneous ODEs.

\textbf{Case 1}: We get distinct real roots, therefore our basis solutions look like
\[
    [y_k = e^{m_k x}]_{k = 1}^n
\]

\textbf{Case 2}: We get a real root $m_k$ of multiplicity $p$, therefore our basis solutions look like
\[
    [y_{k, j} = x^{j - 1}e^{m_k x}]_{j = 1}^p
\]

\textbf{Case 3}: We get a complex conjugate pair of roots in the form $m_k = \alpha \pm i\beta$, therefore our basis solutions look like
\[
    y_{k, 1} = e^{\alpha x}\cos(\beta x) \quad \text{and} \quad y_{k, 2} = e^{\alpha x}\sin(\beta x)
\]

\begin{exercise}
    Solve $y'' + 4y' + 7y = 0$.
\end{exercise}
\begin{solution}
    $y = e^{mx} \iff y' = me^{mx} \iff y'' = m^2e^{mx}$.
    \begin{align*}
        m^2e^{mx} + 4me^{mx} + 7e^{mx} = 0
        &\iff m^2 + 4m + 7 = 0
    \end{align*}
    By the quadratic formula
    \begin{align*}
        m
        &= \frac{-4 \pm \sqrt{16 - 28}}{2}
        = \frac{-4 \pm \sqrt{-12}}{2} \\
        &= \frac{-4 \pm 2i\sqrt{3}}{2}
        = -2 \pm i\sqrt{3}
    \end{align*}
    Therefore $y = c_1 e^{-2x}\cos(\sqrt{3}x) + c_2 e^{-2x}\sin(\sqrt{3}x)$ solves the ODE.
\end{solution}

\begin{exercise}
    Solve $y'' + 9y = 0$.
\end{exercise}
\begin{solution}
    $y = e^{mx} \iff y'' = m^2e^{mx}$.
    \begin{align*}
        m^2e^{mx} + 9e^{mx} = 0
        &\iff m^2 + 9 = 0
    \end{align*}
    For the characteristic polynomial
    \begin{align*}
        0
        &= m^2 + 9
        = (m - 3i)(m + 3i)
    \end{align*}
    Since $m = \pm 3i$, then $y = c_1 \cos(3x) + c_2 \sin(3x)$ solves the ODE.
\end{solution}



\begin{exercise}
    Solve $y''' + 3y'' - 4y = 0$.
\end{exercise}
\begin{solution}
    $y = e^{mx} \iff y'' = m^2e^{mx} \iff y''' = m^3e^{mx}$.
    \begin{align*}
        0 = m^3e^{mx} + 3m^2{e^mx} - 4e^{mx}
        &\iff 0 = m^3 + 3m^2 - 4
    \end{align*}
    Then the characteristic polynomial is $m^3 + 3m^2 - 4$.
    Since $1$ divides $1$ and $1$, $2$, and $4$ divide $4$ then the possible roots are
    \[
        \pm( 2, 4 )
    \]
    By synthetic division $-2$ is a root so
    \begin{align*}
        m^3 + 3m^2 - 4
        &= (m + 2)(m^2 + m - 2) \\
        &= (m + 2)(m - 1)(m + 2) \\
        &= (m + 2)^2(m - 1)
    \end{align*}
    Since $0 = m^3 + 3m^2 - 4 = (m + 2)^2(m - 1)$ means that $m = 1$ or $m - 2$ then
    \[
        y = c_1e^{x} + c_2e^{-2x} + c_3xe^{-2x}
    \]
    solves the ODE.
\end{solution}

\end{document}