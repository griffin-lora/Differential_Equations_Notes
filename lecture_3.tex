\documentclass[notes.tex]{subfiles}

\begin{document}

\setcounter{section}{2}
\section{Lecture (01/15/25)}

\subsection{IVPs and Intervals of Validity}
\begin{exercise}
    Solve the IVP and state the interval of validity of $x^2 y' = y - xy$ with $y|_{x = -1} = -1$
\end{exercise}
\begin{solution}
    \begin{align*}
        x^2 y' = y - xy
        &\iff x^2\frac{dy}{dx} = y - xy = y(1 - x) \\
        &\iff \frac{dy}{dx} = y\left(\frac{1 - x}{x^2}\right)
        \iff dy = y\left(\frac{1 - x}{x^2}\right)dx \\
        &\iff \frac{1}{y}dy = \frac{1 - x}{x^2}dx \\
        &\iff \int \frac{1}{y}dy = \int \frac{1 - x}{x^2}dx = \int \left(\frac{1}{x^2} - \frac{1}{x}\right)dx \\
        &\iff \ln|y| = C - \frac{1}{x} - \ln|x| \\
        &\iff y = e^{C - \frac{1}{x} - \ln|x|} = D\frac{e^{-\frac{1}{x}}}{|x|}
    \end{align*}
    We now have a solution for the ODE, now we need to consider the initial condition so
    \begin{align*}
        -1 = D\frac{e^{-\frac{1}{-1}}}{|-1|} = De &\iff D = -\frac{1}{e}
    \end{align*}
    So now our solution is
    \begin{equation} \label{test_answer_1}
        y = -\frac{e^{-\frac{1}{x}}}{e|x|} = -\frac{e^{-\frac{1}{x} - 1}}{|x|}
    \end{equation}
    Now for the interval of validity, consider that the solution function exists iff $x \neq 0$, and since the solution function is analytic then its derivative also has the same domain. So the interval of validity is $(-\infty, 0)$. Therefore our final solution can be written as
    \begin{equation} \label{test_answer_2}
        y = \frac{e^{-\frac{1}{x} - 1}}{x}
    \end{equation}
\end{solution}

Note that
\begin{enumerate}[label=\arabic*)]
    \item we prefer explicit solutions over implicit solutions where possible; and
    \item either the answer \eqref{test_answer_1} or \eqref{test_answer_2} would be acceptable on tests.
\end{enumerate}

\subsection{Slope Fields}
\begin{definition}[Slope Field]
Given a first order ODE that satisfies the existence and uniqueness theorem in the form
\begin{equation} \tag{$\star$} \label{sfield}
    y' = f(x, y)
\end{equation}
then the \textit{slope field} for \eqref{sfield} is the scalar field generated by $f$.
\end{definition}
The idea is that for any point $(x, y)$, \eqref{sfield} will tell us the slope of the tangent line for a solution curve at the given point.

\begin{exercise}
    Sketch the slope field for the ODE $y' = x + 2y$ with at least 16 slope values. Then sketch a solution curve satisfying $y|_{x = 0} = -1$.
\end{exercise}
% TODO: Actually do this

\subsection{Autonomous ODEs}
\begin{definition}[Autonomous ODE]
    A first order ODE is called \textit{autonomous} iff it can be written in the form $y' = f(y)$.
\end{definition}
Notice that this means that we are free from the influence of $x$. Therefore, every column of the slope field is identical.
% TODO: Insert sketch, what??

\begin{definition}[Critical Point]
    Given an autonomous ODE in the form $y' = f(y)$, then a value $y_0$ is called a \textit{critical point} iff $f(y_0) = 0$
\end{definition}
\begin{lemma} \label{solution_curves_no_cross}
    If $y_0$ is a critical point for an autonomous ODE, then no solution curves for the ODE cross the line $y = y_0$.
\end{lemma}
\begin{proof}
    If a solution curve $y = g(x)$ has some point $x_0$ such that $y_0 = g(x_0)$ then, by definition, $y' = 0$ when $x = x_0$. Since the DE is autonomous, we are independent from the influence of $x$, therefore $y'$ is constant for $x \geq x_0$, so $y' = 0$ for $x \geq x_0$. Therefore, our solution curve cannot change $y$-value for all $x \geq 0$, so it will never cross $y = y_0$.
\end{proof}
\begin{definition}[Stable, Unstable, and Semistable Critical Points]
    Given an autonomous ODE in the form $y' = f(y)$, then a critical point $y_0$ is called
    \begin{enumerate}[label=\arabic*)]
        \item \textit{stable} iff $y'$ attracts the solution curve to $y_0$ for values of $y$ between $y_0$ and the next critical point;
        \item \textit{unstable} iff $y'$ repels the solution curve from $y_0$ for values of $y$ between $y_0$ and the next critical point; and
        \item \textit{semistable} iff $y'$ attracts one side of the solution curve to and repels the other side from $y_0$ for values of $y$ between $y_0$ and the next critical point.
    \end{enumerate}
\end{definition}
\begin{lemma} \label{constant_IVT}
    Given an autonomous ODE in the form $y' = f(y)$ such that $f$ is continuous, then between critical points of the ODE $y'$ will obtain a constant sign.
\end{lemma}
\begin{proof}
    Since $f$ is continuous, this follows trivially from the intermediate value theorem.
\end{proof}
% \begin{lemma}
%     Given an autonomous ODE in the form $y' = f(y)$ such that $f$ is continuous, then all critical points are either stable, unstable, or semistable.
% \end{lemma}

\begin{definition}[Phase Portrait]
    Given an autonomous ODE in the form $y' = f(y)$, the \textit{phase portrait} of the ODE is a single column of the slope field with the critical points and sign of $y'$ marked, with an up arrowhead used for $y' > 0$ and a down arrowhead used for $y' < 0$. Because of \cref{constant_IVT}, we only mark one arrow per section of the phase portrait between critical points.
\end{definition}

\begin{exercise} \label{phase_portrait_ex}
    Sketch the phase portrait for $y' = (y + 1)(y - 2)^2 (y - 5)$
\end{exercise}
\begin{solution}
    The critical points are $-1$, $2$ and $5$. We have a degree four polynomial with all positive factors so we get \\
    \begin{tikzpicture}
        \draw[solid, thick, <->] (0, -3) -- (0, 7);
        \draw[thick] (0.2, -1) -- ++ (-0.4, 0) node[left] {$-1$};
        \draw[thick] (0.2, 2) -- ++ (-0.4, 0) node[left] {$2$};
        \draw[thick] (0.2, 5) -- ++ (-0.4, 0) node[left] {$5$};
        \draw[solid, thick, -{Stealth[scale=2]}] (0, -1.5) -- (0, -1.4);
        \draw[solid, thick, -{Stealth[scale=2]}] (0, 1) -- (0, 0.9);
        \draw[solid, thick, -{Stealth[scale=2]}] (0, 3) -- (0, 2.9);
        \draw[solid, thick, -{Stealth[scale=2]}] (0, 6) -- (0, 6.1);
    \end{tikzpicture} \\
    Therefore $-1$ is stable, $5$ is unstable and $2$ is semistable.
\end{solution}

\begin{exercise}
    Given the phase portrait we found in \cref{phase_portrait_ex} and an IVP in the form $y' = f(y)$ with $y|_{x = 0} = 4$ and a solution curve $y = g(x)$, find $\lim_{x\to\infty} g(x)$.
\end{exercise}
\begin{solution}
    Since $y|_{x = 0} = 4$, then $2 \leq g(x) \leq 5$ by \cref{solution_curves_no_cross}. Since $y'|_{x = 0} < 0$, then $y' < 0$ for $2 \leq y \leq 5$ by \cref{constant_IVT}. Since we are bounded by $2$ and $5$, then $y' < 0$ for our entire solution curve. Therefore, we must asymptotically approach the lower bound, so $\lim_{x\to\infty} g(x) = 2$.
\end{solution}

\end{document}