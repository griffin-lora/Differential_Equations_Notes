\documentclass[notes]{subfiles}
\externaldocument{lecture_16}

\begin{document}

\setcounter{section}{17}
\section{Lecture (4/7/25)}

\subsection{Laplace Transform of Derivatives}
\begin{theorem}
    If $f$ is a $C^n$ function of exponential type then
    \[
        \lap D^n f(s) = s^n(\lap f(s)) - \sum_{i = 1}^n s^{n - i}(D^{i - 1}f(0))
    \]
    for $s \geq 0$
\end{theorem}
\begin{proof}
    For the base case of $n = 0$ consider that
    \begin{align*}
        s^0(\lap f(s)) - \sum_{i = 1}^0 s^{n - i}(D^{i - 1}f(0))
        &= \lap f(s)
    \end{align*}
    We now assume our induction hypothesis and consider
    \begin{align*}
        \lap D^{n + 1} f(s)
        &= \int_0^\infty (D^{n + 1}f(s))e^{-st}dt
        = \lim_{b\to\infty} \int_0^b (D^{n + 1}f(s))e^{-st}dt
    \end{align*}
    Let $u = e^{-st} \implies du = -se^{-st}dt$ and $dv = D^{n + 1}f(s)dt \impliedby v = D^n f(s)$ so
    \begin{align*}
        (\lap D^{n + 1} f)(s)
        &= \int_0^\infty (D^{n + 1}f(s))e^{-st}dt
        = \lim_{b\to\infty} \left( D^n f(s)e^{-st}\Big|_{t = 0}^{t = b} + s\int_0^b (D^n f(s))e^{-st}dt \right) \\
        &= \lim_{b\to\infty} (D^n f(b)e^{-sb} - D^n f(0)) + s\int_0^\infty (D^n f(s))e^{-st}dt \\
        &= -D^n f(0) + s(\lap D^n f(s))
        = s(\lap D^n f(s)) - D^n f(0)
    \end{align*}
    Then by our induction hypothesis
    \begin{align*}
        (\lap D^{n + 1} f)(s)
        &= s(\lap D^n f(s)) - D^n f(0) \\
        &= s(s^n(\lap f(s)) - \sum_{i = 1}^n s^{n - i}(D^{i - 1}f(0))) - D^n f(0) \\
        &= s^{n + 1}(\lap f(s)) - \sum_{i = 1}^n s^{(n + 1) - i}(D^{i - 1}f(0)) - s^0 (D^n f(0)) \\
        &= s^{n + 1}(\lap f(s)) - \sum_{i = 1}^n s^{(n + 1) - i}(D^{i - 1}f(0)) - \sum_{i = n + 1}^{n + 1} s^{(n + 1) - i}(D^{i - 1}f(0)) \\
        &= s^{n + 1}(\lap f(s)) - \sum_{i = 1}^{n + 1} s^{(n + 1) - i}(D^{i - 1}f(0))
    \end{align*}
\end{proof}

\begin{corollary}
    If $f$ is a $C^1$ function of exponential type then
    \[
        \lap f'(s) = s(\lap f(s)) - f(0)
    \]
    for $s \geq 0$
\end{corollary}

\begin{corollary}
    If $f$ is a $C^2$ function of exponential type then
    \[
        \lap f''(s) = s^2(\lap f(s)) - sf(0) - f'(0)
    \]
    for $s \geq 0$
\end{corollary}

\subsection{Step Functions}
\begin{definition}[Heaviside Step Function]
    The \textit{Heaviside step function} $\heav\colon \real \to \{ 0, 1 \}$ is defined by
    \[
        \heav(t) = \begin{cases}
            0 & t < 0 \\
            1 & t \geq 1
        \end{cases}
    \]
\end{definition}
Below is a graph of the Heaviside step function

\begin{center}%
\begin{tikzpicture}
    \draw[<->] (-3, 0) -- (3, 0) node[right]{$t$};
    \draw[<->] (0, -1) -- (0, 3) node[above]{$\heav(t)$};
    
    \draw[thick] (0.2, 2) -- (-0.2, 2) node[left] {$1$};

    \draw[line width = 1pt, draw = blue] (-3, 0) -- (0, 0);
    \draw[line width = 1pt, draw = blue, dashed] (0, 0) -- (0, 2);
    \draw[fill = white, draw = blue] (0, 0) circle (2pt);
    \filldraw[blue] (0, 2) circle (2pt);
    \draw[line width = 1pt, draw = blue, -stealth] (0, 2) -- (3, 2);
\end{tikzpicture}
\end{center}

We can shift the graph to the right by $a$ by considering $\heav(t - a)$
\[
    \heav(t - a) = \begin{cases}
        0 & t < a \\
        1 & t \geq a
    \end{cases}
\]
Which graphs as
\begin{center}%
\begin{tikzpicture}
    \draw[<->] (-3, 0) -- (3, 0) node[right]{$t$};
    \draw[<->] (0, -1) -- (0, 3) node[above]{$\heav(t - a)$};
    
    \draw[thick] (0.2, 2) -- (-0.2, 2) node[left] {$1$};
    \draw[thick] (1, 0) -- (1, 0) node[below] {$a$};

    \draw[line width = 1pt, draw = blue] (-3, 0) -- (1, 0);
    \draw[line width = 1pt, draw = blue, dashed] (1, 0) -- (1, 2);
    \draw[fill = white, draw = blue] (1, 0) circle (2pt);
    \filldraw[blue] (1, 2) circle (2pt);
    \draw[line width = 1pt, draw = blue, -stealth] (1, 2) -- (3, 2);
\end{tikzpicture}
\end{center}

We can subtract two shifted step functions to get $_{a}\Pi_b(t) = \heav(t - a) - \heav(t - b)$ where $a < b$ which graphs as

\begin{center}%
\begin{tikzpicture}
    \draw[<->] (-1, 0) -- (6, 0) node[right]{$t$};
    \draw[<->] (0, -1) -- (0, 3) node[above]{$_{a}\Pi_b(t)$};
    
    \draw[thick] (-0.2, 2) -- (0.2, 2) node[right] {$1$};
    \draw[thick] (2, 0) -- (2, 0) node[below] {$a$};
    \draw[thick] (4, 0) -- (4, 0) node[below] {$b$};

    \draw[line width = 1pt, draw = blue] (0, 0) -- (2, 0);
    \draw[line width = 1pt, draw = blue, dashed] (2, 0) -- (2, 2);
    \draw[fill = white, draw = blue] (2, 0) circle (2pt);
    \filldraw[blue] (2, 2) circle (2pt);
    \draw[line width = 1pt, draw = blue] (2, 2) -- (4, 2);
    \draw[line width = 1pt, draw = blue, dashed] (4, 2) -- (4, 0);
    \draw[fill = white, draw = blue] (4, 2) circle (2pt);
    \filldraw[blue] (4, 0) circle (2pt);
    \draw[line width = 1pt, draw = blue] (4, 0) -- (6, 0);
\end{tikzpicture}
\end{center}

\begin{theorem}
    Every piecewise function can be written as a sum of step functions.
\end{theorem}

\begin{exercise}
    Write 
    \[
        f(t) = \begin{cases}
            t^2 & 0 \leq t < 3 \\
            -2t + 4 & 3 \leq t < 5 \\
            e^{-6t} & t \geq 5
        \end{cases}
    \]
    in terms of step functions.
\end{exercise}
\begin{solution}
    First consider the graph of $f$ so
    \begin{center}%
    \begin{tikzpicture}
        \begin{axis}[
            axis lines = middle,
            smooth,
            xlabel = $t$,
            ylabel = $f(t)$,
            xticklabel = \empty,
            yticklabel = \empty,
            xtick style = {draw= none},
            ytick style = {draw = none},
            xmin = -1,
            xmax = 10,
            ymin = -7,
            ymax = 11
        ]
            \addplot[
                domain = 0:3,
                samples = 25,
                color = blue
            ] {x*x};
            \addplot[
                domain = 3:5,
                samples = 2,
                color = blue
            ] {-2*x + 4};
            \addplot[
                domain = 5:10,
                samples = 25,
                color = blue
            ] {200*e^(-x)};
            \draw[thick] (3, -0.2) -- (3, 0.2) node[below] {$3$};
            \draw[thick] (5, -0.2) -- (5, 0.2) node[below] {$5$};
            \draw[fill = white, draw = blue] (3, 9) circle (2pt);
            \filldraw[blue] (3, -2) circle (2pt);
            \draw[fill = white, draw = blue] (5, -6) circle (2pt);
            \filldraw[blue] (5, 1.34758939982) circle (2pt);
        \end{axis}
    \end{tikzpicture}%
    \end{center}
    Notice that we can break this down into ${t^2} _{0}\Pi_3(t)$, ${(-2t + 4)}_{3}\Pi_5(t)$, and $e^{-6t}\heav(t - 5)$ so
    \begin{align*}
        f(t)
        &= {t^2} _{0}\Pi_3(t) + {(-2t + 4)}_{3}\Pi_5(t) + e^{-6t}\heav(t - 5) \\
        &= t^2(\heav(t) - \heav(t - 3)) + (-2t + 4)(\heav(t - 3) - \heav(t - 5)) + e^{-6t}\heav(t - 5) \\
        &= t^2\heav(t) - t^2\heav(t - 3) - 2t\heav(t - 3) + 2t\heav(t - 5) + 4\heav(t - 3) - 4\heav(t - 5) + e^{-6t}\heav(t - 5) \\
        &= t^2\heav(t) + (-t^2 - 2t + 4)\heav(t - 3) + (2t - 4 + e^{-6t})\heav(t - 5)
    \end{align*}
\end{solution}

\subsection{Using Laplace transforms to solve modeling problems}
\begin{exercise}
    A mass of $1\si{\kg}$ is attatched to a spring where $k = 2$ and is submerged in a fluid where $\beta = 3$. The mass is driven by a force $f(t) = 2\sin(\pi t)$. Find $x$ where $t \mapsto x$ with initial conditions at $t = 0$ of $x = 0$, $x' = 1$ using the method 
    shown in \cref{laplace_solution_diagram}.
\end{exercise}
\begin{solution}
    From \cref{damped_harmonic_oscillator_eq} we have to solve the equation
    \[
        x'' + 3x' + 2x = 2\sin(\pi t)
    \]
    Therefore
    \begin{align*}
        &x'' + 3x' + 2x = 2\sin(\pi t)
        \iff \lap (x'' + 3x' + 2x) = \lap (2\sin(\pi t)) \\
        \iff& s^2 X - sx|_{t = 0} - x'|_{t = 0} + 3(sX - x|_{t = 0}) + 2X = \frac{2\pi}{s^2 + \pi^2} \\
        \iff& s^2X - 1 + 3sX + 2X = \frac{2\pi}{s^2 + \pi^2} \\
        \iff& X(s^2 + 3s + 2) = \frac{2\pi}{s^2 + \pi^2} + 1 = \frac{s^2 + \pi^2 + 2\pi}{s^2 + \pi^2} \\
        \iff& X = \frac{s^2 + \pi^2 + 2\pi}{(s^2 + \pi^2)(s^2 + 3s + 2)}
    \end{align*}
    Now we can do partial fraction decomposition so
    \begin{align*}
        X
        &= \frac{s^2 + \pi^2 + 2\pi}{(s^2 + \pi^2)(s + 1)(s + 2)}
        = \frac{As + B}{s^2 + \pi^2} + \frac{C}{s + 1} + \frac{D}{s + 2} \\
        &= \frac{As}{s^2 + \pi^2} + \frac{B}{s^2 + \pi^2} + \frac{C}{s + 1} + \frac{D}{s + 2}
    \end{align*}
    and find the inverse Laplace transform on both sides so
    \begin{align*}
        x = A\cos(\pi t) + \frac{B}{\pi}\sin(\pi t) + Ce^{-t} + De^{-2t}
    \end{align*}
\end{solution}

\end{document}