\documentclass[notes]{subfiles}

\begin{document}

\setcounter{section}{1}
\section{Lecture (01/13/25)}

\subsection{Ways of writing first order ODEs}

All first order ODEs can be written as
$y' = f(x, y)$ or $P(x, y)dx + Q(x, y)dy = 0$ where $P$ and $Q$ are functions. The LHS of the second equation is the differential $1$-form for the ODE.

\begin{example}
    $(y - x)dx + 4xdy = 0$. If we want to write this in the way of the first equation then
    \begin{align*}
        (y - x)dx + 4xdy = 0
        &\iff 4xdy = -(y - x)dx \\
        &\iff 4x\frac{dy}{dx} = -(y - x) = x - y \\
        &\iff \frac{dy}{dx} = \frac{x - y}{4x} = \frac{1}{4} - \frac{y}{4x}
    \end{align*}
    Now if we want to see that it is linear then notice that
    \[
        \frac{dy}{dx} = \frac{1}{4} - \frac{y}{4x} \iff \frac{dy}{dx} + \frac{y}{4x} = \frac{1}{4}
    \]
    So $a_1(x) = 1$, $a_0(x) = \frac{1}{4x}$ and $g(x) = \frac{1}{4}$.
\end{example}

\subsection{Initial Value Problems}
\begin{definition}[Initial Value Problem]
    An \textsl{initial value problem} (IVP) of order $n$ is an order $n$ ODE together with a set of restrictions called \textsl{initial conditions} which take the form of when $x = x_0$ then $\left[ \frac{d^i y}{dx^i} = y_i \right]_{i = 0}^n$.
\end{definition}
Note that
\begin{enumerate}[label=(\arabic*)]
    \item $x_0, [y_i]_{i = 0}^n$ are constants; and
    \item IVPs are not boundary value problems.
\end{enumerate}

\begin{example}
    $y' = f(x, y)$ and $y|_{x = x_0} = y_0$ is an order one IVP.
\end{example}

\begin{exercise}
    Given that the two-paramter family of functions $y = c_1\cos 2x + c_2\sin 2x$ solve the ODE $y'' + 4y = 0$, find $c_1, c_2$ that solve the IVP where $x = 0 \implies y = 1, y' = 2$.
\end{exercise}
\begin{solution}
    $y' = -2c_1 \sin 2x + 2c_2 \cos 2x$ so consider that if $x = 0$ then
    $1 = y = c_1 \cos (0) + c_2 \sin (0) = c_1$ and $2 = y = -2c_1 \sin (0) + 2c_2 \cos (0) = 2c_2$. Therefore $c_1 = c_2 = 1$ so $y = \cos 2x + \sin 2x$ solves the IVP.
\end{solution}

\begin{theorem}[Existence and Uniqueness for order one IVPs] \label{EUIVP}
    If $S \subseteq \real^2$ and $f\colon S \to \real$ such that there exists $D = [a, b] \times [c, d]$, where $(x_0, y_0) \in D$ and both $f$ and $f_y$ are continuous on $D$, then there exists $I_0 = (x_0 - \varepsilon, x_0 + \varepsilon) \subseteq [a, b]$, $\varepsilon >0$, such that $g\colon I_0 \to \real$ as $y = g(x)$ is the unique solution to the IVP
    \[
        y' = f(x, y) \text{ with } y|_{x = x_0} = y_0
    \]
\end{theorem}

Note that this is sometimes also called Picard's Theorem.

\begin{exercise}
    Does the IVP $y' = x\sqrt{y}$ with $y|_{x = 0} = 0$ satisfy the hypotheses of \cref{EUIVP}?
\end{exercise}
\begin{proof}
    Since $f(x, y) = x\sqrt{y}$ then $f_y(x, y) = \frac{x}{2\sqrt{y}}$. Notice that $(0, 0) \notin \dom f_y$ so we do not satisfy the hypotheses of \cref{EUIVP}. Therefore the IVP is not guaranteed to have a unique solution.
\end{proof}

\subsection{Intervals of Validity}
\begin{definition}[Interval of Validity]
    Given an order $n$ ODE and a function $g$ where $y = g(x)$ solves the ODE then an \textsl{interval of validity} for $g$ is the largest (in the context of the problem) open interval $I$ where $\left[ \frac{d^i y}{dx^i} \right]_{i = 0}^n$ exist and satisfy the ODE. 
\end{definition}

Note that for IVPs we take the largest interval containing the initial input $x_0$

\begin{example}
    Let $y = \frac{1}{x}$ solve an order two ODE. Then consider that $y' = -\frac{1}{x^2}$ and $y'' = \frac{2}{x^3}$ so the intervals of validity for the ODE are $(-\infty, 0)$ and $(0, \infty)$.
\end{example}

\subsection{Separability}
\begin{definition}[Separability]
    A first order ODE $y' = f(x, y)$ is called \textsl{separable} iff we can factor $f$ into the form $f(x, y) = g(x)h(y)$.
\end{definition}

\begin{example}
    $y' = xe^{-y}$ is separable
\end{example}
\begin{example}
    $y' = \sqrt{x - 2y}$ is not separable
\end{example}

\begin{lemma}
    A first order ODE is separable iff it can be written in the form $f(x)dx = g(y)dy$.
\end{lemma}

\begin{procedure}[Solving separable ODEs] ~\par
    \begin{enumerate}[label=(\arabic*)]
        \item Write $y'$ as $\frac{dy}{dx}$.
        \item Multiply both sides by $dx$.
        \item Get $x$ with $dx$ and $y$ with $dy$.
        \item Integrate both sides.
    \end{enumerate}
\end{procedure}

\begin{exercise}
    Solve the ODE $y' = xe^{-y}$.
\end{exercise}
\begin{solution}
    The ODE is separable so consider
    \begin{align*}
        y' = xe^{-y}
        &\iff \frac{dy}{dx} = xe^{-y}
        \iff dy = xe^{-y}dx
        \iff \frac{dy}{e^{-y}} = xdx \\
        &\iff e^y dy = xdx
        \iff \int e^y dy = \int xdx \\
        &\iff e^y = C + \frac{1}{2}x^2
    \end{align*}
    We now have an implicit solution for the ODE. Now we need to get an explicit solution for the ODE so
    \[
        e^y = C + \frac{1}{2}x^2 \iff y = \ln\left(C + \frac{1}{2}x^2\right)
    \]
\end{solution}

\end{document}