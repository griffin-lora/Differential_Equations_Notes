\documentclass[notes]{subfiles}

\begin{document}

\setcounter{section}{3}
\section{Textbook Notes}
\subsection{Exact Equations}

\begin{definition}[Differential of a two-variate function]
    Given $S \subseteq \mathbb{R}^2$ and a differentiable function $f\colon S \to \mathbb{R}$, then the \textit{differential} on $f$ at any point $(x, y) \in S$ is defined as
    \[
        df(x, y) = f_x(x, y)dx + f_y(x, y)dy
    \]
\end{definition}

\begin{definition}[Exact Differential]
    Given $S \subseteq \mathbb{R}^2$, then a differential form $P(x, y)dx + Q(x, y)dy$ defined on $S$ is an \textit{exact differential} iff there exists some $f\colon S \to \mathbb{R}$ such that $df(x, y) = P(x, y)dx + Q(x, y)dy$.
\end{definition}

\begin{definition}[Exact Differential Equation] ~\par
    An ODE in the form $P(x, y)dx + Q(x, y)dy = 0$ is an \textit{exact differential equation} iff the LHS is an exact differential.
\end{definition}

\begin{example}
    $x^2 y^3 dx + x^3 y^2 dy = 0$ is an exact equation since the LHS is an exact differential form by $d\left(\frac{1}{3}x^3 y^3\right) = x^2 y^3 dx + x^3 y^2 dy$
\end{example}

\begin{theorem} \label{greens_theorem}
    Given a rectangular region $R \subseteq \mathbb{R}^2$ and functions $P, Q\colon R \to \mathbb{R}$ that have continuous partial derivatives, then $P(x, y)dx + Q(x, y)dy$ is an exact differential iff $Q_x(x, y) = P_y(x, y)$
\end{theorem}
\begin{proof}
    For the first direction, we are given that $P(x, y)dx + Q(x, y)dy$ is an exact differential, then consider that there exists $f\colon R \to \mathbb{R}$ such that $df(x, y) = P(x, y)dx + Q(x, y)dy$. By Clairaut's theorem $f_{xy}(x, y) = f_{yx}(x, y)$. Now consider that
    \begin{align*}
        Q(x, y) = f_y(x, y)
        &\iff Q_x(x, y) = f_{yx}(x, y) = f_{xy}(x, y)
    \end{align*}
    and
    \begin{align*}
        P(x, y) = f_x(x, y)
        &\iff P_y(x, y) = f_{xy}(x, y)
    \end{align*}
    Now notice that
    \begin{align*}
        Q_x(x, y) = f_{xy}(x, y) = P_y(x, y)
    \end{align*}
    % For the second direction, we are given that $Q_x(x, y) - P_y(x, y) = 0$, therefore $Q_x(x, y) = P_y(x, y)$. Now consider some $f$ such that $f_x(x, y) = P(x, y)$, and notice that
    % \begin{align*}
    %     f_x(x, y) = P(x, y)
    %     &\iff f(x, y) = g(y) + \int P(x, y)dx
    % \end{align*}
    % Now consider that
    % \begin{align*}
    %     f_y(x, y)
    %     &= g'(y) + \frac{\partial}{\partial y} \int P(x, y)dx
    %     = g'(y) + \int P_y(x, y)dx \\
    %     &= g'(y) + \int Q_x(x, y)dx
    %     = g'(y) + h(y) + Q(x, y)
    % \end{align*}
\end{proof}

\begin{exercise}
    Is $\atan y dx + x dy = 0$ an exact equation?
\end{exercise}
\begin{proof}
    Consider
    \begin{align*}
        Q_x(x, y)
        &= \frac{\partial}{\partial x} x
        = 1
        \neq P_y(x, y)
        = \frac{\partial}{\partial y} \atan y
        = \frac{1}{1 + y^2}
    \end{align*}
    Therefore, by \cref{greens_theorem}, we do not have an exact equation
\end{proof}

\begin{exercise}
    Is $x^2 y^3 dx + x^3 y^2 dy = 0$ an exact equation?
\end{exercise}
\begin{proof}
    Consider
    \begin{align*}
        Q_x(x, y)
        &= \frac{\partial}{\partial x} (x^3 y^2)
        = 3 x^2 y^2
        = P_y(x, y)
        = \frac{\partial}{\partial y} (x^2 y^3)
        = 3 x^2 y^2
    \end{align*}
    Therefore, by \cref{greens_theorem}, we have an exact equation
\end{proof}

\begin{lemma} \label{exact_eq_sol}
    Given an exact equation $P(x, y)dx + Q(x, y)dy = 0$ and $f$ such that $df(x, y) = P(x, y)dx + Q(x, y)dy$ then $f(x, y) = C$ is a family of solutions to the exact equation.
\end{lemma}
\begin{proof}
    \begin{align*}
        f(x, y) = C
        &\iff df(x, y) = dC
        \iff P(x, y)dx + Q(x, y)dy = 0
    \end{align*}
\end{proof}

From \cref{exact_eq_sol}, we can conclude that solutions to exact equations will be level curves of the potential function $f$.

\subsection{Solving Exact Equations}
\begin{procedure}[Solving exact equations] ~\par
    \begin{enumerate}[label=(\arabic*)]
        \item We know there exists $f$ such that \par
        \begin{tabularx}{\linewidth}{X X X}
            \begin{equation} \tag{$1$} \label{f_x_ident}
                f_x = P
            \end{equation}
            & \[ \text{and} \] & % big hack here but i dont care
            \begin{equation} \tag{$2$} \label{f_y_ident}
                f_y = Q
            \end{equation}
        \end{tabularx}
        Furthermore, $f(x, y) = C$ will be a solution.
        \item Recover $f$ by integrating either $f_x$ or $f_y$. The result of that will include an unknown single variable function.
        \item Recover the unknown function by differentiating $f$ with respect to the unknown function's dependent variable and using either \eqref{f_x_ident} or \eqref{f_y_ident} appropriately.
    \end{enumerate}
\end{procedure}

\begin{exercise}
    Solve $2xydx + (x^2 - 1)dy = 0$.
\end{exercise}
\begin{solution}
    First, consider that
    \begin{align*}
        Q_x(x, y) = \frac{\partial}{\partial x}(x^2 - 1) = 2x = P_y(x, y) = \frac{\partial}{\partial y}(2xy) = 2x
    \end{align*}
    Therefore, we have an exact equation by \cref{greens_theorem}. So now we know there exists some $f$ such that $df(x, y) = 2xydx + (x^2 - 1)dy$. Now consider that $f_x(x, y) = 2xy$ so we can say $f(x, y) = g(y) + \int 2xydx = g(y) + x^2 y$. Now consider that
    \begin{align*}
        x^2 - 1 = f_y(x, y) = g'(y) + x^2
        &\iff -1 = g'(y)
        \iff g(y) = C - y
    \end{align*}
    Therefore we have
    \begin{align*}
        f(x, y) = C - y + x^2 y
    \end{align*}
    Therefore, $C - y + x^2 y = 0$ is a solution to the exact equation by \cref{exact_eq_sol}.
\end{solution}

\begin{exercise} \label{ee_exercise_1}
    Solve $(e^{2y} - y\cos(xy))dx + (2xe^{2y} - x\cos(xy) + 2y)dy = 0$
\end{exercise}
\begin{solution}
    First, consider that
    \begin{align*}
        \frac{\partial}{\partial x} (2xe^{2y} - x\cos(xy) + 2y)
        &= 2e^{2y} - \cos(xy) + xy\sin(xy) \\
        &= \frac{\partial}{\partial y} (e^{2y} - y\cos(xy)) \\
        &= 2e^{2y} - \cos(xy) + xy\sin(xy)
    \end{align*}
    Since we have an exact equation, let $f$ such that $f_x(x, y) = e^{2y} - y\cos(xy)$ and consider
    \begin{align*}
        &f_x(x, y) = e^{2y} - y\cos(xy)
        \iff f(x, y) = g(y) + x e^{2y} - \sin(xy) \\
        &\iff 2xe^{2y} - x\cos(xy) + 2y = f_y(x, y) = g'(y) + 2x e^{2y} - x\cos(xy) \\
        &\iff g'(y) = 2y
        \iff g(y) = C + y^2 \\
        &\iff f(x, y) = C + y^2 + x e^{2y} - \sin(xy)
    \end{align*}
    Therefore, $C + y^2 + x e^{2y} - \sin(xy) = 0$ solves the exact equation.
\end{solution}

\begin{exercise}
    Solve $(e^{2y} - y\cos(xy))dx + (2xe^{2y} - x\cos(xy) + 2y)dy = 0$ by first using $f_y(x, y) = 2xe^{2y} - x\cos(xy) + 2y$.
\end{exercise}
\begin{solution}
    We already know this is an exact equation by \cref{ee_exercise_1} so let $f$ such that $f_y(x, y) = 2xe^{2y} - x\cos(xy) + 2y$ and consider
    \begin{align*}
        &f_y(x, y) = 2xe^{2y} - x\cos(xy) + 2y
        \iff f(x, y) = g(x) + xe^{2y} - \sin(xy) + y^2 \\
        &\iff e^{2y} - y\cos(xy) = f_x(x, y) = g'(x) + e^{2y} - y\cos(xy) \\
        &\iff g'(x) = 0
        \iff g(x) = C \\
        &\iff f(x, y) = C + xe^{2y} - \sin(xy) + y^2
    \end{align*}
    Therefore, $C + xe^{2y} - \sin(xy) + y^2 = 0$ solves the exact equation as seen in \cref{ee_exercise_1}.
\end{solution}

\begin{exercise}
    Solve $\frac{dy}{dx} = \frac{xy^2 - \cos x \sin x}{y(1 - x^2)}$ where when $x = 0$ that $y = 2$.
\end{exercise}
\begin{solution}
    First, we need to rewrite the equation to have a differential form on the LHS and $0$ on the right side so
    \begin{align*}
        \frac{dy}{dx} = \frac{xy^2 - \cos x \sin x}{y(1 - x^2)}
        &\iff dy = \frac{xy^2 - \cos x\sin x}{y(1 - x^2)}dx \\
        &\iff y(1 - x^2)dy = (xy^2 - \cos x\sin x)dx \\
        &\iff (\cos x\sin x - xy^2)dx + y(1 - x^2)dy = 0 \\
        &\iff (\cos x\sin x - xy^2)dx + (y - x^2 y)dy = 0
    \end{align*}
    Now consider that
    \begin{align*}
        \frac{\partial}{\partial x} (y - x^2 y)
        &= -2xy
        = \frac{\partial}{\partial y}(\cos x\sin x - xy^2) = -2xy
    \end{align*}
    Therefore, we let $f$ such that $f_x(x, y) = \cos x\sin x - xy^2$ then consider
    \begin{align*}
        &f_x(x, y) = \cos x\sin x - xy^2
        \iff f(x, y) = g(y) - \frac{1}{2}\cos^2 x - \frac{1}{2} x^2 y^2 \\
        &\iff y - x^2 y = f_y(x, y) = g'(y) - x^2 y \\
        &\iff g'(y) = y
        \iff g(y) = C + \frac{1}{2}y^2 \\
        &\iff f(x, y) = C + \frac{1}{2}y^2 - \frac{1}{2}\cos^2 x - \frac{1}{2} x^2 y^2
    \end{align*}
    So our general solution to the exact equation is $C + \frac{1}{2}y^2 - \frac{1}{2}\cos^2 x - \frac{1}{2} x^2 y^2 = 0$. \\
    An explicit solution is possible here so we will try to make this equation as implicit as possible before considering the IVP. So consider
    \begin{align*}
        C + \frac{1}{2}y^2 - \frac{1}{2}\cos^2 x - \frac{1}{2} x^2 y^2 = 0
        &\iff \frac{1}{2}y^2 - \frac{1}{2} x^2 y^2 = \frac{1}{2}\cos^2 x - C \\
        &\iff y^2 - x^2 y^2 = \cos^2 x - 2C \\
        &\iff y^2(1 - x^2) = \cos^2 x - 2C \\
        &\iff y^2 = \frac{\cos^2 x - 2C}{1 - x^2} = \frac{\cos^2 x + D}{1 - x^2}
    \end{align*}
    For the IVP we consider
    \begin{align*}
        x = 0
        &\iff 2^2 = 4 = \frac{\cos^2 (0) + D}{1 - (0)^2} = 1 + D
        \iff D = 3
    \end{align*}
    Therefore, the implicit solution to the IVP is $y^2 = \frac{\cos^2 x + 3}{1 - x^2}$.
    Since when $x = 0$, $y = 2$, then we take the principal square root of the solution to get an explicit solution to the IVP as follows
    \begin{align*}
        y^2 = \frac{\cos^2 x + 3}{1 - x^2}
        &\iff y = \sqrt{\frac{\cos^2 x + 3}{1 - x^2}}
    \end{align*}
\end{solution}

\begin{exercise}
    Solve $(\sin y - y\sin x)dx + \left(\cos x + x\cos y + \frac{y}{y^2 - y - 2}\right)dy = 0$.
\end{exercise}
\begin{solution}
    First consider that
    \begin{align*}
        \frac{\partial}{\partial x}\left(\cos x + x\cos y + \frac{y}{y^2 - y - 2}\right)
        &= -\sin x + \cos y \\
        &= \frac{\partial}{\partial y}(\sin y - y\sin x) \\
        &= \cos y - \sin x
    \end{align*}
    Now let $f$ such that $f_x(x, y) = \sin y - y\sin x$ and consider
    \begin{align*}
        &f_x(x, y) = \sin y - y\sin x
        \iff f(x, y) = g(y) + x\sin y + y\cos x \\
        &\iff \cos x + x\cos y + \frac{y}{y^2 - y - 2} = f_y(x, y) = g'(y) + x\cos y + \cos x \\
        &\iff g'(y) = \frac{y}{y^2 - y - 2} \iff g(y) = \int \frac{y}{y^2 - y - 2} dy
    \end{align*}
    This integral is not trivial, we will now use partial fraction decomposition so
    \begin{align*}
        g(y)
        &= \int \frac{y}{y^2 - y - 2} dy
        = \int \frac{y}{(y - 2)(y + 1)}dy
        = \int \left( \frac{A}{y - 2} + \frac{B}{y + 1} \right)dy
    \end{align*}
    Now we need to find the values of $A$ and $B$ so
    \begin{align*}
        A(y + 1) + B(y - 2) = y
        &\iff Ay + A + By - 2B = y \\
        &\iff (A + B)y + (A - 2B) = y \\
        &\iff A + B = 1
        \iff A = 1 - B \\
        &\iff 0 = A - 2B = 1 - B - 2B = 1 - 3B \\
        &\iff 1 = 3B
        \iff B = \frac{1}{3} \iff A = \frac{2}{3}
    \end{align*}
    Now we can consider the integral again so
    \begin{align*}
        g(y)
        &= \int \left( \frac{A}{y - 2} + \frac{B}{y + 1} \right)dy
        = \int \left( \frac{2}{3}\left( \frac{1}{y - 2} \right) + \frac{1}{3}\left(\frac{1}{y + 1}\right) \right) dy \\
        &= C + \frac{2}{3}\ln|y - 2| + \frac{1}{3}\ln|y + 1|
    \end{align*}
    Therefore
    \begin{align*}
        f(x, y) = g(y) + x\sin y + y\cos x = C + \frac{2}{3}\ln|y - 2| + \frac{1}{3}\ln|y + 1| + x\sin y + y\cos x
    \end{align*}
    So $C + \frac{2}{3}\ln|y - 2| + \frac{1}{3}\ln|y + 1| + x\sin y + y\cos x = 0$ is the solution to the exact equation.
\end{solution}

\end{document}