\documentclass[notes]{subfiles}

\begin{document}

\setcounter{section}{8}
\section{Lecture (2/17/25)}
\subsection{Linear Independence}
Note that linear independence for function spaces is expressed as $\{ f_i \}_{i = 1}^n$ is linearly independent on $I$.

\begin{exercise} \label{triv_lin_ind_exer}
    Let $f_1(x) = 2$, $f_2(x) = 6 + 3x$, and $f_3(x) = x$. Is $\{ f_1, f_2, f_3 \}$ linearly independent on $(-\infty, \infty)$?
\end{exercise}
\begin{proof}
    Consider that $f_2 = 3f_1 + 3f_3$, therefore $f_2 \in \lspan\{ f_1, f_3 \}$ so $\{ f_1, f_2, f_3 \}$ is not linearly independent for any $I$. Therefore it is not linearly independent on $(-\infty, \infty)$.
\end{proof}

However, a simple check like in \cref{triv_lin_ind_exer} is not always possible. In that case we need a specific object to check linear independence.

\begin{definition}[Wro\'nskian]
    Given $[f_j]_{j = 1}^n$ are functions of class $D^{n - 1}$ then the \textit{Wro\'nskian} of those functions is defined as follows
    \[
        W(f_j)_{j = 1}^n = \det(D^{i - 1}f_j)_{i = 1, j = 1}^{n, n}
    \]
\end{definition}

\begin{theorem} \label{wronskian_theorem}
    If $[f_j]_{j = 1}^n$ are functions with a defined Wro\'nskian then $(W(f_j)_{j = 1}^n)(x) \neq 0$ for any $x \in I$ iff $\{ f_j \}_{j = 1}^n$ is linearly independent on $I$.
\end{theorem}

\begin{exercise}
    Let $f_1(x) = e^x$, $f_2(x) = e^{2x}$, and $f_3(x) = e^{3x}$. Is $\{ f_1, f_2, f_3 \}$ linearly independent on $(-\infty, \infty)$?
\end{exercise}
\begin{proof}
    We can't immediately tell which functions are in the span of the others. However, we can consider
    \[
        0 = c_1e^x + c_2e^{2x} + c_3e^{3x}
    \]
    But also notice that we can't make any particular conclusions here. In this case we should use \cref{wronskian_theorem} so
    \begin{align*}
        (W(f_j)_{j = 1}^n)(x)
        &= (\det(D^{i - 1}f_j)_{i = 1, j = 1}^{n, n})(x) \\
        &= \begin{vmatrix}
            f_1 & f_2 & f_3 \\
            Df_1 & Df_2 & Df_3 \\
            D^2f_1 & D^2f_2 & D^2f_3
        \end{vmatrix}(x) \\
        &= \begin{vmatrix}
            e^x & e^{2x} & e^{3x} \\
            e^x & 2e^{2x} & 3e^{3x} \\
            e^x & 4e^{2x} & 9e^{3x}
        \end{vmatrix}
    \end{align*}
    Now we can evaluate at $x = 0$ so
    \begin{align*}
        (W(f_j)_{j = 1}^n)(0)
        &= \begin{vmatrix}
            e^0 & e^{2(0)} & e^{3(0)} \\
            e^0 & 2e^{2(0)} & 3e^{3(0)} \\
            e^0 & 4e^{2(0)} & 9e^{3(0)}
        \end{vmatrix}
        = \begin{vmatrix}
            1 & 1 & 1 \\
            1 & 2 & 3 \\
            1 & 4 & 9
        \end{vmatrix} \\
        &= \begin{vmatrix}
            2 & 3 \\
            4 & 9
        \end{vmatrix}
        - \begin{vmatrix}
            1 & 3 \\
            1 & 9
        \end{vmatrix}
        + \begin{vmatrix}
            1 & 2 \\
            1 & 4
        \end{vmatrix} \\
        &= 6 - 6 + 2 = 2 \neq 0
    \end{align*}
    Since $0 \in (-\infty, \infty)$ then by \cref{wronskian_theorem} $\{ f_1, f_2, f_3 \}$ are linearly independent on $(-\infty, \infty)$.
\end{proof}

\subsection{Solutions to Homogeneous Linear ODEs}
To solve homogeneous linear ODEs of order $n$ of the form $Lf = 0$ we can use the following theorem
\begin{theorem}
    Given a homogeneous linear ODE of order $n$, then for the associated linear differential operator $L$, $\dim \nul L = n$.
\end{theorem}
Therefore the dimension of the solution space is precisely the order of the ODE.

\begin{definition}[Particular Solution]
    Given a linear ODE in the form $Lf = g$, then $f_p$ is the \textit{particular solution} for the ODE iff $f_p \notin \nul L$ such that $L(f_p) = g$.
\end{definition}

% Not sure if this is the case
% \begin{theorem} \label{particular_solution_exists}
%     Given a linear ODE, then the ODE has a particular solution $f_p$. 
% \end{theorem}

\begin{lemma}
    Given a homogeneous linear ODE of order $n$ in the form $Lf = g$, a particular solution $f_p$, and $f_0 = f_n + f_p$ where $f_n \in \nul L$, then $L(f_0) = g$.
\end{lemma}
\begin{proof}
    \begin{align*}
        L(f_0)
        &= L(f_n + f_p)
        = L(f_n) + L(f_p)
        = 0 + g
        = g
    \end{align*}
\end{proof}

\end{document}