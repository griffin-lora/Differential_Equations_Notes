\documentclass[notes]{subfiles}

\begin{document}

\setcounter{section}{9}
\section{Lecture (2/19/25)}
\subsection{Reduction of Order}
The idea here is that if we are given an order two homogeneous linear ODE, a basis solution $y_1$, then we can find the basis solution $y_2$ such that the general solution is $y = c_1y_1 + c_2y_2$.

We are given an order two homogeneous linear ODE in the form
\[
    a_2(x)y'' + a_1(x)y' + a_0(x)y = 0
\]
such that $a_2(x) \neq 0$ for all $x \in I$. We are given $y_1$ as a basis solution.

We will now rewrite as
\begin{align*}
    a_2(x)y'' + a_1(x)y' + a_0(x)y = 0
    &\iff y'' + \frac{a_1(x)}{a_2(x)}y' + \frac{a_0(x)}{a_2(x)}y = 0 \\
    &\iff y'' + P(x)y' + Q(x)y = 0
\end{align*}
Now the trick here is to let $y_2 = uy_1$ therefore
\begin{align*}
    y_2 = uy_1
    &\iff y_2' = u'y_1 + uy_1' \\
    &\iff y_2'' = u''y_1 + u'y_1' + u'y_1' + uy_1'' = u''y_1 + 2u'y_1' + uy_1'' % haha you are a binomial
\end{align*}

Now we plug $y_2$ in as a solution so

\begin{align*}
    0
    &= y_2'' + P(x)y_2' + Q(x)y_2 \\
    &= (u''y_1 + 2u'y_1' + uy_1'') + P(x)(u'y_1 + uy_1') + Q(x)(uy_1) \\
    &= u(y_1'' + P(x)y_1' + y_1) + u''y_1 + 2u'y_1' + P(x)(u'y_1) \\
    &= u''y_1 + 2u'y_1' + P(x)(u'y_1) \\
    &= y_1u'' + (2y_1' + P(x)y_1)u'
\end{align*}

To get this in the form of a linear ODE we make the substitution $w = u'$ solution

\begin{align*}
    y_1w' + (2y_1' + P(x)y_1)w = 0
    &\iff y_1\frac{dw}{dx} = -(2y_1' + P(x)y_1)w \\
    &\iff y_1dw = -(2y_1' + P(x)y_1)wdx \\
    &\iff \frac{1}{w}dw = -\frac{2y_1' + P(x)y_1}{y_1}dx = \left( -\frac{2y_1'}{y_1} - P(x) \right)dx \\
    &\iff \int \frac{1}{w}dw = -\int \frac{2y_1'}{y_1}dx - \int P(x)dx \\
    &\iff \ln w = -2\ln y_1 - \int P(x)dx \\
    &\iff w = e^{-2\ln y_1}e^{-\int P(x)dx} = \frac{e^{-\int P(x)dx}}{y_1^2} \\
    &\iff u = \int \frac{e^{-\int P(x)dx}}{y_1^2} dx
\end{align*}

Therefore we can now express $y_2$ as
\begin{equation} \label{reduction_order_formula}
    y_2 = y_1\int \frac{e^{-\int P(x)dx}}{y_1^2} dx
\end{equation}
This formula should be memorized.

\begin{exercise}
    Consider that $y_1 = x^2$ solves the ODE $x^2y'' - 3xy' + 4y = 0$ on $x \in (0, \infty)$. Find the general solution to the ODE.
\end{exercise}
\begin{solution}
    Since $x \neq 0$ then
    \begin{align*}
        x^2y'' - 3xy' + 4y = 0
        &\iff y'' - \frac{3}{x}y' + \frac{4}{x^2}y = 0
    \end{align*}
    Now we consider by \cref{reduction_order_formula}
    \begin{align*}
        y_2
        &= y_1\int \frac{e^{-\int P(x)dx}}{y_1^2} dx
        = x^2\int \frac{e^{\int \frac{3}{x}dx}}{x^4} dx \\
        &= x^2\int \frac{e^{3\ln x}}{x^4} dx
        = x^2\int \frac{x^3}{x^4} dx
        = x^2\int \frac{1}{x} dx \\
        &= x^2\ln x
    \end{align*}
    Therefore $y = c_1x^2 + c_2x^2\ln x$ solves the ODE most generally.
\end{solution}

\begin{exercise} \label{constant_coeff_example}
    Find the general solution to $y'' - 3y' + 2y = 5e^{3x}$ using that a basis solution to $y'' - 3y' + 2y = 0$ is $y_1 = e^x$.
\end{exercise}
\begin{solution}
    Consider that the solution will be in the form $y = y_n + y_p$ where $y_p$ is the particular solution, therefore we first need to find $y_2$ for the homogeneous ODE.
    \begin{align*}
        y_2
        &= y_1\int \frac{e^{-\int P(x)dx}}{y_1^2} dx
        = e^x\int \frac{e^{\int 3dx}}{e^{2x}} dx \\
        &= e^x\int \frac{e^{3x}}{e^{2x}} dx
        = e^x\int e^x dx
        = e^{2x}
    \end{align*}
    So $y_n = c_1e^x + c_2e^{2x}$. Intuitively we can see that the particular solution must be of the form $y_p = Ae^{3x}$. Therefore
    \begin{align*}
        y_p' = 3Ae^{3x} && y_p'' = 9Ae^{3x}
    \end{align*}
    Now we can plug this into the equation so
    \begin{align*}
        9Ae^{3x} - 9Ae^{3x} + 2Ae^{3x} = 5e^{3x}
        &\iff 2Ae^{3x} = 5e^{3x} \\
        &\iff 2A = 5
        \iff A = \frac{5}{2}
    \end{align*}
    Therefore $y_p = \frac{5}{2}e^{3x}$ so the general solution is $y = c_1e^x + c_2e^{3x} + \frac{5}{2}e^{3x}$.
\end{solution}

\subsection{Digression on Constant Coefficients}
Notice that we ``guessed" the form of $y_p$ in \cref{constant_coeff_example}. This is not strictly necessary given certain conditions. If we let $g(x) = 5e^{3x}$ then we the associated linear differential operator equation for \cref{constant_coeff_example} is
\begin{equation} \tag{$\star$} \label{linear_differential_op_eq}
    D^2 - 3D + 2 = g
\end{equation}
We know how to solve homogeneous linear ODEs, so if we can find some $L_0$ such that $L_0(g) = 0$ then we can apply it to both sides of the equation. Finding $L_0$ is now analogous to finding what homogenous linear differential equation $g$ solves.

Consider $L_0 = D - 3$ then
\begin{align*}
    L_0(g)
    &= (D - 3)(g)
    = g' - 3g
\end{align*}
and
\begin{align*}
    (g' - 3g)(x)
    &= 15e^{3x} - 15e^{3x}
    = 0
\end{align*}
Therefore $L_0(g) = 0$ so we can consider multiplying both sides of \eqref{linear_differential_op_eq} by $L_0$ so
\begin{align*}
    (L_0)(D^2 - 3D + 2) = L_0(g)
    &\iff (D - 3)(D^2 - 3D + 2) = 0 \\
    &\iff D^3 - 3D^2 + 2D - 3D^2 + 9D - 6 = 0 \\
    &\iff D^3 - 6D^2 + 11D - 6 = 0
\end{align*}

We now could solve this as an order three homogenous linear ODE to get the particular solution.

This is the general idea for solving all constant coefficient linear ODEs of order $n$.

\end{document}